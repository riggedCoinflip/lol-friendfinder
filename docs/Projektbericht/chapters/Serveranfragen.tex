


//TO COMPLETE...
- GraphQL
- React-Bootsrap

3x Mögliche Alternativen für die API-Anfragen im React

Rest
GraphQL
Ein Endpunkt…
Queries, Mutations und Subscriptions…

Kein Over-Fetching:_
Mit GraphQL werden nur die Daten zurückgegeben, die benötigt werden, und nicht mehr.
Nur bestimmte zugelassen Datentypen werden akzeptiert.

3x Warum sollte Apollo Client und GraphQL in Betracht gezogen werden?
https://www.howtographql.com/basics/1-graphql-is-the-better-rest/

https://www.freecodecamp.org/news/react-apollo-client-2020-tutorial/

GraphQL (gql)
- API Requests -> Queries and mutations

\textbf{Wie wurden die Anfragen in unserem Projekt verwaltet?}\\
Alle Queries/Abfragen und Mutations wurden in einem Ordner gesammelt. 
Diese wurden für die spätere Verwendung in den React-Komponenten exportiert.

\begin{lstlisting}
    export const GET_MY_INFO = gql`
    {
      userSelf {
        _id
        name
        aboutMe
        languages
        gender
        avatar
        ingameRole
        dateOfBirth
        friends { user chat }        
        blocked
      }
    }
  `
//Beispiel einer Abfrage
\end{lstlisting}


\newpage
\textbf{Apollo Client}\\
Nachdem eine Abfrage exportiert wurde, ist sie bereit, in einer React-Komponente importiert und angewendet zu werden.

\begin{lstlisting}
import { GET_MY_INFO } from "./GraphQL/Queries"

const { loading, error, data } = useQuery(
GET_MY_INFO,
ContextHeader(token),
)
\end{lstlisting}
Die Konstante ContextHeader enthält das Token in der Struktur, die erforderlich ist, um die Abfrage nur dann stellen zu können, wenn der Benutzer dazu berechtigt ist.
Sollte das Token undefined, null oder ungültig sein, wird der Server ein Fehler zurückgegeben.

Der useQuery-Hook liefert ein Ergebnisobjekt, das Folgendes enthält:

\textbf{loading:}\\
Ein boolescher Wert, der angibt, ob die Abfrage in Bearbeitung ist.
Wenn loading true ist, ist die Anfrage noch nicht abgeschlossen. Typischerweise kann diese Information verwendet werden, um einen Lade-Spinner anzuzeigen.
\\

\textbf{error:}\\
Ein Laufzeitfehler mit den Eigenschaften graphQLErrors und networkError.
Enthält Informationen darüber, was bei Ihrer Abfrage schief gelaufen ist.
\\

\textbf{date:}\\
Ein Objekt, das das Ergebnis der GraphQL-Abfrage enthält.
Es enthält die tatsächlichen Daten vom Server. 
\\


\textbf{Axios} \\

Was ist axios…

Über axios wurde eine Post-Anfrage bereitgestellt, die ermöglicht hat Bilder auf die S3 Speicher von AWS hochzuladen.

In der Benutzeroberfläche wurde ein Eingabefeld "input" verwendet mit type = "file" definiert.
Um die Art der hochzuladenden Dateien einzuschränken, wurde eine Reihe zulässiger Dateiformate definiert, die an den Server gesendet werden dürfen.
\\

\begin{lstlisting}
    const admittedImageFormats = ["png", "jpg", "jpeg"]
\end{lstlisting}
\newline

Die Größe der hochzuladenden Datei wurde um 1Mb abgegrenzt.
Durch die Eigenschaft „size“ der ausgewählten Datei konnten wir auf die Größe der Datei zugreifen.

\begin{lstlisting}
    const imageSize = e.target.files[0].size
\end{lstlisting}