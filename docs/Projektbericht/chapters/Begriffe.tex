%%% MUSTER %%%
%\newglossaryentry{}
%{name=, description={}}

%%%%%%%%%%%%
%%% TIMO %%%
%%%%%%%%%%%%

\newglossaryentry{MVP}
{
    name=MVP,
    description={Minimum Viable Product, erste funktionsfähige Version des Produkts; dient dazu zu prüfen, ob es Sinn macht, das Produkt weiterzuentwickeln und ggf. anzupassen}
}


\newglossaryentry{Responsive Webdesign}
{
    name=Responsive Webdesign,
    description={Die Webseite wird so aufgebaut, dass sie auf möglichst vielen Endgeräten mit verschiedenen Bildschirmgrößen und Eingabemethoden (Maus, Touchscreen) skaliert wird und bedienbar ist.}
}

\newglossaryentry{DQL}
{
    name=DQL,
    description={Data Query Language, Datenabfragesprache für SQL}
}

\newglossaryentry{Tinder}
{
    name=Tinder,
    description={Datingportal, bei der Nutzer in Kontakt treten, indem sie sich bei der Kontaktsuche gegenseitig einen \enquote{like} geben}
}

\newglossaryentry{Brute-Force-Angriff}
{
    name=Brute-Force-Angriff,
    description={Versuch, ein Passwort zu knacken, indem jede mögliche Kombination nacheinander ausprobiert wird.}
}

\newglossaryentry{UI/UX}
{
    name=UI/UX,
    description={User Interface / User Experience. Benutzeroberfläche / Benutzerfreundlichkeit.} 
}

\newglossaryentry{MERN}
{
    name=MERN,
    description={Beliebter Techstack für das Web, bei dem MongoDB, ExpressJS, React und NodeJS verwendet werden}
}

\newglossaryentry{MongoDB}
{
    name=MongoDB,
    description={Beliebte Dokumentbasierte Datenbank}
}

\newglossaryentry{NodeJS}
{
    name=NodeJS,
    description={JS-Laufzeit, die es ermöglicht, JS auf dem Server auszuführen}
}

\newglossaryentry{Express}
{
    name=Express,
    description={Framework für NodeJS, welches die Entwicklung von Webanwendungen vereinfacht}
}

\newglossaryentry{React}
{
    name=React,
    description={JS UI-Framework für Webseiten}
}

\newglossaryentry{JavaScript}
{
    name=JavaScript,
    description={Beliebte Scriptsprache, die vor allem für Webseiten verwendet wird. Neben Web-Assembly die einzige Sprache, welche sich auf Klienten-Seite ausführen lässt. Nach ECMAScript standardisiert}
}

\newglossaryentry{REST-API}
{
    name=REST-API,
    description={REpresentional State Transfer; Architekturstil, der eine Schnittstelle über die Protokolloptionen POST (Create), GET (Read),  PUT/PATCH (Update) und DELETE (Delete) nach RFC-2616 ermöglicht}
}

\newglossaryentry{AWS S3}
{
    name=AWS S3,
    description={Amazon Simple Storage Service; Objektspeicher über Webschnittstelle}
}

\newglossaryentry{Multer}
{
    name=Multer,
    description={Middleware für NodeJS zum behandeln von POST-Anfragen des Typs \enquote{multipart/form-data}}
}

\newglossaryentry{Apollo}
{
    name=Apollo,
    description={Beliebte Implementation der GraphQl-Spezifikation}
}

\newglossaryentry{Mongoose}
{
    name=Mongoose,
    description={NodeJS-Paket zur Objektmodellierung von MongoDB}
}

\newglossaryentry{NoSQL}
{
    name=NoSQL,
    description={Not only SQL; Sammelbegriff für Datenbanken, die einen nicht-relationalen Ansatz verfolgen.}
}

\newglossaryentry{ORDBMS}
{
    name=,
    description={ObjektRelationales DatenBank ManagementSysten}
}

\newglossaryentry{ACID}
{
    name=ACID,
    description={Atomicity, Consistency, Isolation, Durability; häufig erwünschte Eigenschaften von Transaktionen von Datenbanken}
}

\newglossaryentry{JSON}
{
    name=JSON,
    description={JavaScipt Object Notation; Datenformat zum Austausch von Daten zwischen Anwendungen}
}

\newglossaryentry{Sharding}
{
    name=Sharding,
    description={Methode der Datenbankpartitionierung, Aufteilung der Datenbank in mehrere \enquote{Scherben}, die zusammen alle Daten verwalten}
}

\newglossaryentry{Replikatgruppe}
{
    name=Replikatgruppe,
    description={Gruppe von Datenbankprozessen, welche die selben Daten verwalten. Dadurch wird Redundanz und Hocherreichbarkeit gewahrt.}
}

\newglossaryentry{Constraint}
{
    name=Constraints,
    description={Bedingung, die eine Variable erfüllen muss, um in die Datenbank aufgenommen zu werden.}
}

\newglossaryentry{Multi-Master-Systeme}
{
    name=Multi-Master-Systeme,
    description={Datenbanksysteme, welche mehrere Prozesse ermöglichen, Schreibzugriffe durchzuführen.}
}

\newglossaryentry{Write-Ahead-Log}
{
    name=Write-Ahead-Log,
    description={Protokoll, welches vor jeder Datenänderung der PostgreSQL-Datenbank erstellt wird. Dies verringert die nötige Anzahl an Schribzugriffen und erleichtert das Erstellen und Aufspielen von Backups}
}

\newglossaryentry{Salt}
{
    name=,
    description={Wert, der dem zu hashenden Wert vor dem Hashen hinzugefügt wird, um die Wahrscheinlichkeit gleicher Eingabewerte zu verringern und damit die Informationssicherung zu erhöhen.}
}

\newglossaryentry{DDOS-Angriff}
{
    name=DDOS-Angriff,
    description={Distributed Denial Of Service Attack; Angriff eines Webdienstes durch Überlastung durch etliche Anfragen von einer Vielzahl von Geräten}
}

\newglossaryentry{DBaaS}
{
    name=DBaaS,
    description={DataBase as a Service; Datenbankinfrastruktur wird vom Anbieter aus der Cloud bezogen, statt diese selbst zu betreuen}
}

\newglossaryentry{Dev-Prod-Vergleichbarkeit}
{
    name=Dev-Prod-Vergleichbarkeit,
    description={Regel 10 der 12-Faktor-App, Entwicklungs- und Produktionsumgebung so ähnlich wie möglich halten, um die Probleme durch Inkompatibilitäten zu vermeiden}
}

\newglossaryentry{npm}
{
    name=npm,
    description={node package manager; Bibliothek und Paketmanager für Pakete der NodeJS-Runtime}
}

\newglossaryentry{RFC-7519}
{
    name=RFC-7519,
    description={Norm, auf der JSON Web Tokens basieren}
}

\newglossaryentry{JSON Web Token}
{
    name=JSON Web Token,
    description={Nach RFC-7519 genormter Zugriffsschlüssel, durch den die Identität des Nutzers einer Webseite gespeichert wird}
}

\newglossaryentry{API}
{
    name=API,
    description={Application Programming Interface; Programmierschnittstelle, durch die verschiedene Programme Daten austauschen können}
}

\newglossaryentry{Mutation}
{
    name=Mutation,
    description={Veränderung der bestehenden Daten nach GraphQl-Spezifikation, entspricht POST/PUT/PATCH/DELETE in REST-API}
}

\newglossaryentry{Angular}
{
    name=Angular,
    description={Angular ist ein Framework von Google für die Entwicklung von Mobil- und Webanwendungen. } 
}

\newglossaryentry{TypeScript}
{
    name=TypeScript,
    description={Superset von JS, wird zur Laufzeit nach JS transpiliert, bietet im Gegensatz zu JS starke Typisierung an, dadurch wird eine Quelle für Softwarefehler vermieden}
}

\newglossaryentry{ECMAScript 2015}
{
    name=ECMAScript 2015,
    description={ES2015, früher ES6; Eine Version von ECMAScript, die viele neue Funktionen bietet, JS basiert auf ECMAScript}
}

\newglossaryentry{XSS-Angriff}
{
    name=XSS-Angriff,
    description={Cross-Side-Scripting; Art der HTML-Injection, Kann auftreten, wenn Nutzereingaben mit Schadcode ungefiltert vom Server akzeptiert und ausgeführt werden}
}

\newglossaryentry{StackOverFlow}
{
    name=StackOverFlow,
    description={Beliebtes Programmierer-Forum}
}

\newglossaryentry{Github}
{
    name=Github,
    description={Beliebter Dienst zur Versionsverwaltung für Entwicklungsprojekte.}
}

\newglossaryentry{HTML-DOM}
{
    name=HTML-DOM,
    description={Das HTML-DOM ist ein Standardobjektmodell und eine Programmierschnittstelle für HTML.} 
}

\newglossaryentry{Jest}
{
    name=Jest,
    description={Beliebtes JS Testing-Framework. Stand 2022.}
}

\newglossaryentry{Cypress}
{
    name=Cypress,
    description={Cypress ist ein End-to-End-Test-Framework für die Automatisierung von Web-Tests.} 
}

\newglossaryentry{OpenBase}
{
    name=OpenBase,
    description={OpenBase hilft Entwicklern bei der Auswahl aus Millionen von Open-Source-Paketen} 
}

\newglossaryentry{Framework}
{
    name=Framework,
    description={Liefert einen Rahmen mit verschiedenen Funktionen, die für das Schreiben einer Anwendung verwendet und selektiv angepasst werden können.}
}

\newglossaryentry{atomare Tests}
{
    name=atomare Tests,
    description={Softwartests, welche in sich abgeschlossen sind und andere Tests nicht beeinflussen.}
}

\newglossaryentry{UTF-8}
{
    name=UTF-8,
    description={Weit verbreitete Kodierung für Unicode-Zeichen, sehr beliebt für Webseiten}
}

\newglossaryentry{DQL}
{
    name=DQL,
    description={Data Query Language, Datenabfragesprache für SQL}
}

%PRINTING BEGRIFFE
\textbf{ACID}\\
Atomicity, Consistency, Isolation, Durability; häufig erwünschte Eigenschaften von Transaktionen von Datenbanken.
\\\\
\textbf{atomare Tests}\\
Softwartests, welche in sich abgeschlossen sind und andere Tests nicht beeinflussen.
\\\\
\textbf{Angular}\\
Angular ist ein Framework von Google für die Entwicklung von Mobil- und Webanwendungen.
\\\\
\textbf{Apollo}\\
Beliebte Implementation der GraphQl-Spezifikation.
\\\\
\textbf{API}\\
Application Programming Interface; Programmierschnittstelle, durch die verschiedene Programme Daten austauschen können.
\\\\
\textbf{AWS S3}\\
Amazon Simple Storage Service; Objektspeicher über Webschnittstelle.
\\\\
\textbf{DQL}\\
Data Query Language, Datenabfragesprache für SQL
\\\\
\textbf{Destrukturierende Zuweisung}\\
Die destrukturierende Zuweisung ermöglicht es, Daten aus Arrays oder Objekten zu extrahieren, und zwar mit Hilfe einer Syntax, die der Konstruktion von Array- und Objekt-Literalen nachempfunden ist.
%\url{https//developer.mozilla.org/de/docs/Web/JavaScript/Reference/Operators/Destructuring_assignment}
\\\\
\textbf{MVP}\\
Minimum Viable Product, erste funktionsfähige Version des Produkts; dient dazu zu prüfen, ob es Sinn macht, das Produkt weiterzuentwickeln und ggf. anzupassen.
\\\\
\textbf{JSX}\\
Es heißt JSX und ist eine Syntaxerweiterung für JavaScript. JSX erinnert vielleicht an eine Template-Sprache, aber es verfügt über die volle Leistungsfähigkeit von JavaScript. JSX erzeugt React-"Elemente"
\\\\
\textbf{Over-Fetching}\\
Empfang von überschüssigen Daten durch eine Abfrage.
%Reference https//jwt.io/
\\\\
\textbf{undefined}\\
Eine Variable, der kein Wert zugewiesen wurde oder die überhaupt nicht deklariert wurde (nicht deklariert, existiert nicht), ist undefiniert. Eine Methode oder Anweisung gibt auch undefiniert zurück, wenn der ausgewerteten Variablen kein Wert zugewiesen wurde. Eine Funktion gibt undefiniert zurück, wenn kein Wert zurückgegeben wurde.
\\\\
\textbf{Responsive Webdesign}\\
Die Webseite wird so aufgebaut, dass sie auf möglichst vielen Endgeräten mit verschiedenen Bildschirmgrößen und Eingabemethoden (Maus, Touchscreen) skaliert wird und bedienbar ist.
\\\\
\textbf{UTF-8}\\
Weit verbreitete Kodierung für Unicode-Zeichen, sehr beliebt für Webseiten.
\\\\
\textbf{Web Token}\\
JSON-Web-Tokens sind eine dem Industriestandard RFC 7519 entsprechende Methode zur sicheren Darstellung von Forderungen zwischen zwei Parteien.
\\\\
\textbf{Tinder}\\
Datingportal, bei der Nutzer in Kontakt treten, indem sie sich bei der Kontaktsuche gegenseitig einen like geben.
\\\\
\textbf{Brute-Force-Angriff}\\
Versuch, ein Passwort zu knacken, indem jede mögliche Kombination nacheinander ausprobiert wird.
\\\\
\textbf{UI/UX}\\
User Interface / User Experience. Benutzeroberfläche / Benutzerfreundlichkeit.
\\\\
\textbf{MERN}\\
Beliebter Techstack für das Web, bei dem MongoDB, ExpressJS, React und NodeJS verwendet werden.
\\\\
\textbf{MongoDB}\\
Beliebte Dokumentbasierte Datenbank.
\\\\
\textbf{NodeJS}\\
JS-Laufzeit, die es ermöglicht, JS auf dem Server auszuführen.
\\\\
\textbf{Express}\\
Framework für NodeJS, welches die Entwicklung von Webanwendungen vereinfacht.
\\\\
\textbf{React}\\
JS UI-Framework für Webseiten.
\\\\
\textbf{JavaScript}\\
Beliebte Scriptsprache, die vor allem für Webseiten verwendet wird. Neben Web-Assembly die einzige Sprache, welche sich auf Klienten-Seite ausführen lässt. Nach ECMAScript standardisiert.
\\\\
\textbf{REST-API}\\
REpresentional State Transfer; Architekturstil, der eine Schnittstelle über die Protokolloptionen POST (Create), GET (Read),  PUT/PATCH (Update) und DELETE (Delete) nach RFC-2616 ermöglicht.
\\\\
\textbf{Multer}\\
Middleware für NodeJS zum behandeln von POST-Anfragen des Typs multipart/form-data.
\\\\
\textbf{Mongoose}\\
NodeJS-Paket zur Objektmodellierung von MongoDB.
\\\\
\textbf{NoSQL}\\
Not only SQL; Sammelbegriff für Datenbanken, die einen nicht-relationalen Ansatz verfolgen.
\\\\
\textbf{ORDBMS}\\
ObjektRelationales DatenBank ManagementSysten.
\\\\
\textbf{JSON}\\
JavaScipt Object Notation; Datenformat zum Austausch von Daten zwischen Anwendungen.
\\\\
\textbf{Sharding}\\
Methode der Datenbankpartitionierung, Aufteilung der Datenbank in mehrere Scherben die zusammen alle Daten verwalten.
\\\\
\textbf{Replikatgruppe}\\
Gruppe von Datenbankprozessen, welche die selben Daten verwalten. Dadurch wird Redundanz und Hocherreichbarkeit gewahrt.
\\\\
\textbf{Constraint}\\
Bedingung, die eine Variable erfüllen muss, um in die Datenbank aufgenommen zu werden.
\\\\
\textbf{Multi-Master-Systeme}\\
Datenbanksysteme, welche mehrere Prozesse ermöglichen, Schreibzugriffe durchzuführen.
\\\\
\textbf{Write-Ahead-Log}\\
Protokoll, welches vor jeder Datenänderung der PostgreSQL-Datenbank erstellt wird. Dies verringert die nötige Anzahl an Schribzugriffen und erleichtert das Erstellen und Aufspielen von Backups.
\\\\
\textbf{Salt}\\
Wert, der dem zu hashenden Wert vor dem Hashen hinzugefügt wird, um die Wahrscheinlichkeit gleicher Eingabewerte zu verringern und damit die Informationssicherung zu erhöhen.
\\\\
\textbf{DDOS-Angriff}\\
Distributed Denial Of Service Attack; Angriff eines Webdienstes durch Überlastung durch etliche Anfragen von einer Vielzahl von Geräten.
\\\\
\textbf{DBaaS}\\
DataBase as a Service; Datenbankinfrastruktur wird vom Anbieter aus der Cloud bezogen, statt diese selbst zu betreuen.
\\\\
\textbf{Dev-Prod-Vergleichbarkeit}\\
Regel 10 der 12-Faktor-App, Entwicklungs- und Produktionsumgebung so ähnlich wie möglich halten, um die Probleme durch Inkompatibilitäten zu vermeiden.
\\\\
\textbf{npm}\\
node package manager; Bibliothek und Paketmanager für Pakete der NodeJS-Runtime.
\\\\
\textbf{RFC-7519}\\
Norm, auf der JSON Web Tokens basieren.
\\\\
\textbf{JSON Web Token}\\
Nach RFC-7519 genormter Zugriffsschlüssel, durch den die Identität des Nutzers einer Webseite gespeichert wird.
\\\\
\textbf{Mutation}\\
Veränderung der bestehenden Daten nach GraphQl-Spezifikation, entspricht POST/PUT/PATCH/DELETE in REST-API.
\\\\
\textbf{TypeScript}\\
Superset von JS, wird zur Laufzeit nach JS transpiliert, bietet im Gegensatz zu JS starke Typisierung an, dadurch wird eine Quelle für Softwarefehler vermieden.
\\\\
\textbf{ECMAScript 2015}\\
ES2015, früher ES6; Eine Version von ECMAScript, die viele neue Funktionen bietet, JS basiert auf ECMAScript.
\\\\
\textbf{XSS-Angriff}\\
Cross-Side-Scripting; Art der HTML-Injection, Kann auftreten, wenn Nutzereingaben mit Schadcode ungefiltert vom Server akzeptiert und ausgeführt werden.
\\\\
\textbf{StackOverFlow}\\
Beliebtes Programmierer-Forum.
\\\\
\textbf{Github}\\
Beliebter Dienst zur Versionsverwaltung für Entwicklungsprojekte.
\\\\
\textbf{HTML-DOM}\\
Das HTML-DOM ist ein Standardobjektmodell und eine Programmierschnittstelle für HTML.
\\\\
\textbf{Jest}\\
Beliebtes JS Testing-Framework. Stand 2022.
\\\\
\textbf{Cypress}\\
Cypress ist ein End-to-End-Test-Framework für die Automatisierung von Web-Tests.
\\\\
\textbf{OpenBase}\\
OpenBase hilft Entwicklern bei der Auswahl aus Millionen von Open-Source-Paketen.
\\\\
\textbf{Framework}\\
Liefert einen Rahmen mit verschiedenen Funktionen, die für das Schreiben einer Anwendung verwendet und selektiv angepasst werden können.
\\\\

\section*{Akronyme:}
{AWS}: {Amazon Web Services}\\
{DOM}: {Document Object Model}\\
{API}: {Application Programming Interface}\\
{AJAX}: {Asynchronous JavaScript And XML}\\