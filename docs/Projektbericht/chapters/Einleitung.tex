%\subsection*{Einführung in das Thema }%(Motivation, zentrale Begriffe etc)
%\addcontentsline{toc}{subsection}{Einführung in das Thema}
%\subsection*{Hinführung zu den Ergebnissen}
%\addcontentsline{toc}{subsection}{Hinführung zu den Ergebnissen} 
%\subsection*{Ggf. Angabe des Schwerpunktes}
%\addcontentsline{toc}{subsection}{Angabe des Schwerpunktes} 
%\subsection*{Ggf. Einschränkungen darlegen}
%\addcontentsline{toc}{subsection}{Einschränkungen darlegen} 
%\subsection*{Problemstellung}
%\addcontentsline{toc}{subsection}{Problemstellung} 
%\subsection*{Zielstellung der Arbeit}
%\addcontentsline{toc}{subsection}{Zielstellung der Arbeit} 
%\subsection*{Fragestellung der Arbeit}
%\addcontentsline{toc}{subsection}{Fragestellung der Arbeit} 
%\subsubsection*{Struktur der Arbeit}
%\addcontentsline{toc}{subsection}{Struktur der Arbeit} 
Dieser Projektbericht ist in folgenden Kapitel unterteilt:\\
In \textbf{Kapitel \ref{kap_anforderungen}} werden das Projekt in allgemeiner Form sowie seine Anforderungen beschrieben.
\\\\
In \textbf{Kapitel \ref{kap_Technologieinfrastruktur}} wird kurz in der unterschlichen Technologien und die Architektur der Applikation eingegangen.
\\\\
Das \textbf{Kapitel \ref{kap_Datenbank}} befasst sich mit der Gegenüberstellung für die Auswahl zwischen PostgreSQL und MongoDB als Datenbank. Außerdem wird in dem gleichen Kapitel das Datenbankschema erläutert.
\\\\
Das \textbf{Kapitel \ref{kap_Backend}} beschäftigt sich mit den Elementen, die die Verbindung zwischen dem Frontend und der Datenbank ermöglichen. Außerdem wird erklärt, wie die Authentifizierung und Autorisierung mit Hilfe des JWT funktioniert.
\\\\
In \textbf{Kapitel \ref{kap_Schnittstelle}} wird näher auf die Abfragen eingehen, die von der Frontend an die Datenbanken mittels GraphQL erfolgen.
\\\\
%Anhand von zwei Beispielen wird gezeigt, wie Lese- und Schreibabfragen mit Hilfe von GraphQL und ApolloClient durchgeführt wurden.
In \textbf{Kapitel \ref{kap_Frontend}} wird erläutert, welche JavaScript-Frameworks zu Beginn des Projekts berücksichtigt wurden. Schließlich wird gezeigt, wie die Daten für den Endbenutzer visualisiert werden.
\\\\
%Qualitätssicherung
Die Relevanz von Testfällen in Softwareprojekten wird in \textbf{Kapitel \ref{kap_QS}} dargestellt. Es werden ebenfalls die Vorteile automatisierter und eingebetteter Testfällen in der Entwicklungsumgebung besprochen.

