% Was ist das Problem
\paragraph{}
-- Warum Plattform? --

\paragraph{}
Unsere Aufgabe ist es, eine Webplattform zu erschaffen, die es Nutzern ermöglicht, mit anderen Nutzern in Kontakt zu treten und sich im Chat näher kennenzulernen – mit Fokus auf das gemeinsame Spielen von „League of Legends“. Um dies zu erreichen, sind verschiedene Dinge nötig.

\paragraph{}
Nutzer sollen ein Benutzerkonto anlegen können. Auf dem Profil können sie Informationen von sich preisgeben: Name, Alter, einen Freitext, ein Profilbild, sowie präferierte Rollen im Videospiel. Um neue Kontakte kennenzulernen, soll der Nutzer andere Nutzer „mögen“ können. Dazu gibt es eine spezielle Suchfunktion, bei der zufällige andere Nutzer angezeigt werden. In der Suchfunktion ist es möglich, seine Suche mit Filtern einzugrenzen, um nur bestimmte Nutzergruppen zu zeigen.

\paragraph{}
Nutzer erfahren nicht, wer sie mag. Stattdessen werden andere Nutzer, die sie mögen, bei der Suche weiter vorne angezeigt. Wenn beide Nutzer sich gegenseitig mögen, werden sie Freunde und können in Zukunft miteinander kommunizieren.

\paragraph{}
Wie bei jeder Online-Community wird es Nutzer geben, die durch Fehlverhalten negativ auffallen. Es soll Nutzern möglich sein, diese zu blockieren oder sogar zu melden. Im Falle einer Meldung soll der Nutzer besonders unter die Lupe genommen werden und entsprechend mit ihm verfahren werden – bis zur permanenten Sperrung des Nutzerkontos.

\paragraph{}
Benutzer sollen in der Lage sein, Einstellungen zu verwalten – zum Beispiel, ob und wie sie über neue Nachrichten / „Matches“ erfahren sollen. In den Einstellungen sollen sie zudem ihren Account löschen können.

\paragraph{}
Um Nutzer für unsere Plattform zu gewinnen, soll es eine ansprechende Startseite geben, die dem Nutzer ein Bild von unserer Webseite gibt.