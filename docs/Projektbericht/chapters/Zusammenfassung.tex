%Es hat sich gezeigt, dass mit modernen JavaScript-Entwicklungswerkzeugen 
%auch ohne umfassende Kenntnisse der Softwareentwicklung zugänglich sind.
Es wurde gezeigt, wie die Entwicklung einer Webanwendung mit den Technologien des MERN-Stacks erfolgt.
Die Anwendung wurde im Zeitraum von Anfang Mai 2021 bis Mitte August 2021 entwickelt.
Mit grundlegenden Programmierkenntnissen war es möglich, eine funktionelle Anwendung mit einigen Funktionen zu entwicklen.
Diese sind: die Registrierung für neue Benutzer, die Anmeldung für bestehende Benutzer, die Verwaltung derer persönlicher Daten, die Interaktion mit anderen Benutzern auf der Grundlage ihrer Präferenzen und ein auf Textnachrichten basierter Kommunikationskanal.
\\\\
%TInfra
In Kapitel \ref{kap_Technologieinfrastruktur} wurde ein Einblick in die Architektur der Anwendung gegeben. %DB
Die Auswahl der Datenbank und derer Schemata wurden in Kapitel \ref{kap_Datenbank} erläutert.
In den Kapiteln \ref{kap_Backend} und \ref{kap_Schnittstelle} wurde die Logik erläutert, die benötigt ist, um Daten in der Datenbank auszulesen und zu manipulieren.
Dies geschieht durch eine REST-Schnittstelle für das in Amazon S3 gespeicherte Profilbild und GraphQL-Endpunkte für die Verwaltung von Nutzerdaten. 
%Backend %Schnittstelle 
%Frontend
In Kapitel \ref{kap_Frontend} wird gezeigt, welche Kriterien für die Wahl von React berücksichtigt wurden. 
Außerdem wurde die Entwicklung der Benutzeroberfläche mit React-Hooks gezeigt. 
%Qualitätssicherung
Die Erstellung der Unit- und End-To-End-Tests, wie in Kapitel \ref{kap_QS} beschrieben, würde die Testzeit für komplexere Anwendungen verringern.
Im Falle des vorliegenden Projektes handelte es sich um ein zusätzliches Werkzeug, dessen Einbindung in den Entwicklungszyklus etwa fünf Tage in Anspruch nahm.
Durch besagte Tests konnten einige Fehler bereits in der Entwicklungszeit korrigiert werden und die Code-Qualität stark erhöht werden.
Bei der Weiterführung des Projekts erhöht sich mit wachsender Komplexität die Wichtigkeit von Testfällen noch weiter.
%Einer signifikante Wert haben die Unit-Test, 
%Was konnte nicht geschafft werden?
\\\\
Der Fokus des Projekts lag darin, eine funktionelle Minimalanwendung zu entwicklen. Es wurde auf Funktionen verzichtet, welche den geplanten Zeitraum des Projektes überschreiten würden.
Für die Weiterentwicklung des Projektes gibt es einige Funktionen, welche sinnvollerweise in das Projekt integriert werden sollten.
Eine Startseite soll einem potenziellen Nutzer einen Einblick in die Anwendung geben.
Um den Einstieg für neue Nutzer noch weiter zu vereinfachen, bietet es sich an, den Login direkt mit einem Facebook- oder Google-Konto zu ermöglichen.
Für eine bessere Gebrauchstauglichkeit %Usability
bietet es sich an, neben der Webapplikation eine mobile Applikation zu entwickeln.
Auf Basis der verwendeten Technologien würde sich hierfür React-Native eignen.
Um die Echtheit der Nutzer zu prüfen bietet es sich an, eine Kontovalidierung über die Schnittstelle von \enquote{Riot Games} (Entwickler von \enquote{League of Legends}) einzurichten.
Zum Melden von Nutzern besteht bereits eine Schnittstelle, diese wird vom Frontend jedoch noch nicht angesprochen.
Auch soll ein Algorithmus auf Basis der Meldungen entscheiden, welche Benutzer vom Dienst ausgeschlossen werden.
Es sollten Überlegungen zur Finanzierung aufgestellt werden.
Ein kostenpflichtiges Premium-Abo mit zusätzlichen Funktionen und/oder Bannerwerbung bietet sich an.

\section*{Arbeitsverteilung}

\begin{figure}[h!]
    \centering
    \includegraphics[scale=0.7]{sources/Arbeitsverteilung}
    \caption{Arbeitsverteilung des Projekts. Eigene Darstellung.}
    \label{fig:Antwort_Serverabfrage} 
  \end{figure}