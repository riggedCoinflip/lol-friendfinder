% Dokumentation für Praxisprojekt                                           %%
%% (TH Köln -Campus Gummersbach, Fak. 10)                                    %%
%%                                                                           %% %%
%% LIZENZ:                                                                   %%
%% Diese Vorlage darf nicht kommerziell verbreitet                           %%
%% werden. Eine nicht-kommerzielle Weitergabe ist                            %%
%% gestattet.                                                                %%
%%                                                                           %%
%% Von Ludger Schönfeld, M. Sc.,
%% 2014-2017                            %%
%%%%%%%%%%%%%%%%%%%%%%%%%%%%%%%%%%%%%%%%%%%%%%%%%%%%%%%%%%%%%%%%%%%%%%%%%%%%%%%
%%%%%%%%%%%%%%%%%%%%%%%%%%%%%%%%%%%%%%%%%%%%%
%% HEADER                                  %%
%%%%%%%%%%%%%%%%%%%%%%%%%%%%%%%%%%%%%%%%%%%%%
\documentclass[a4paper,12pt,oneside]{article}
% Optionen:
% - a4paper => DIN A4-Format
% - 12pt    => Schriftgröße (weitere  
%              grundlegende Fontgrößen: 10pt, 11pt)
% - oneside => Einseitiger Druck
\setlength{\parindent}{0pt}

%% Verwendete Pakete:
\usepackage[ngerman]{babel} % für die deutsche Sprache
\usepackage{caption} % Für schönere Bildunterschriften
\usepackage[T1]{fontenc} % Schriftkodierung (Für Sonderzeichen u.a.)
\usepackage[utf8]{inputenc} % Für die direkte Eingabe von Umlauten im Editor u.a.
\usepackage{fancyhdr} % Für Kopf- und Fußzeilen
\usepackage{lscape} % Für Querformat

%Mehrere Bilder auf einmal
\usepackage{subfig}
\usepackage{csquotes} % Für sprachangepasste Anführungszeichen


%% Show code in a better way
\usepackage{listings}
\usepackage{xcolor}

\definecolor{dkgreen}{rgb}{0,0.6,0}
\definecolor{gray}{rgb}{0.5,0.5,0.5}
\definecolor{mauve}{rgb}{0.58,0,0.82}

\lstset{frame=tb,
  language=Java,
  aboveskip=3mm,
  belowskip=3mm,
  showstringspaces=false,
  columns=flexible,
  basicstyle={\small\ttfamily},
  numbers=none,
  numberstyle=\tiny\color{gray},
  keywordstyle=\color{blue},
  commentstyle=\color{green},
  stringstyle=\color{purple},
  breaklines=true,
  breakatwhitespace=true,
  tabsize=3
}
%% Schriften (Beispiele)
%% Weitere LaTeX-Schriften im "LaTeX Font Catalogue"
%% unter: http://www.tug.dk/FontCatalogue/.
%% ACHTUNG: Ggf. müssen Schriften noch installiert 
%% werden!

% Serifen-Schriften:
\usepackage{lmodern} % Schriftart "Latin Modern"
%\usepackage{garamond} % Schriftart "Garamond"

%Sans Serif-Schriften:
%\usepackage[scaled]{uarial}
%\usepackage[scaled]{helvet}
%%--------------
\usepackage[normalem]{ulem} % Für das Unterstreichen von Text z.B. mit \uline{}
\usepackage[left=3cm,right=2cm,top=1.5cm,bottom=1cm,
textheight=245mm,textwidth=160mm,includeheadfoot,headsep=1cm,
footskip=1cm,headheight=14.599pt]{geometry} % Einrichtung der Seite 

\usepackage{graphicx} % Zum Laden von Graphiken
\graphicspath{ {./sources/} }

% INFO: Graphiken einbinden
%
% \includegraphics[scale=1.00]{dateiname}
%
% => Ausgabeformat: PDF-Dokument:
%    Es können die folgenden (Graphik-)formate eingebunden
%    werden: .jpg, .png, .pdf, .mps
% 
% => Ausgabeformat: DVI/PS:
%    Folgende (Graphik-)formate werden unterstützt:
%    .eps, .ps, .bmp, .pict, .pntg
\usepackage{epstopdf}

% Pakete für Tabellen
\usepackage{tabularx} % Einfache Tabellen
\usepackage{longtable} % Tabellen als Gleitobjekte (für die Aufteilung bei langen 
 %Tabellen über mehrere Seiten)
\usepackage{multirow} % Für das Verbinden von Zeilen innerhalb einer Tabelle mit
 % \multirow{anzahl}{*}{Text}

% (Zusatz-)Pakete für Formeln
\usepackage{amsmath}
\usepackage{amsthm}
\usepackage{amsfonts}

\usepackage{setspace} % Paket zum Setzen des Zeilenabstandes
% INFO: Zeilenabstand setzen:
%
% Befehle:
% - \singlespacing  => 1-zeilig (Standard)
% - \onehalfspacing => 1,5-zeilig
% - \doublespacing  => 2-zeilig 
\onehalfspacing % Zeilenabstand auf 1,5-zeilig setzen

% Farbboxen (für die Merkkästen in dieser Vorlage):
\usepackage{tcolorbox}
\tcbset{colback=white,colframe=orange,
        fonttitle=\bfseries}

\usepackage[colorlinks,pdfpagelabels,pdfstartview=FitH,
bookmarksopen=true,bookmarksnumbered=true,linkcolor=black,
plainpages=false,hypertexnames=false,citecolor=black]{hyperref} % Für Verlinkungen

%Glossary
\usepackage[xindy,acronym,toc]{glossaries}
\makeglossaries
%\loadglsentries{chapters/Begriffe}
%Cite
\usepackage[style=ieee]{biblatex}
\bibliography{refs} % Entries are in the refs.bib file

%%%%%%%%%%%%%%%%%%%%%%%%%%%%%%%%%%%%%%%%%%%%%
%% DOKUMENT                                %%
%%%%%%%%%%%%%%%%%%%%%%%%%%%%%%%%%%%%%%%%%%%%%
\begin{document}
% Unbeschriftetes Vorblatt (Leere Seite)
\pagestyle{empty} % Seite ohne Kopf- und Fußzeilen
\newpage % Neue Seite
\input{leereSeite} % Ausgelagerte LaTeX-Datei (hier: leereSeite.tex) einbinden

%\newpage
% Deckblatt
\pagestyle{empty}
\begin{titlepage}
  \includegraphics[scale=0.20]{sources/TH_Koeln_Logo}\\
  \begin{center}
    \Large
    Technische Hochschule Köln\\
    Fakultät für Informatik und Ingenieurwissenschaften\\
    \hrule\par\rule{0pt}{2cm}
    \LARGE
    \textsc{BERICHT ZUM PRAXISPROJEKT}\\
    \vspace{0.8cm}
    \huge
    MVP für eine Webanwendung zur Spielersuche\\ 
    \Large
    Eine mit Javascript entwickelte Plattform \\
    \vspace{0.8cm}
    \large
    Vorgelegt an der TH Köln Campus Gummersbach\\
    im Studiengang Wirtschaftsinformatik\\
    \vspace{0.8cm}
    ausgearbeitet von:\\
    \textsc{Timo Kalter} 11127585\\
    \textsc{Carlo Menjivar} 11117929\\
    \vspace{1cm}
    \begin{tabular}{ll} % Einfache Tabelle ohne Rahmen, mit 2 Spalten erzeugen
      \textbf{Erster Prüfer:}  & Prof. Dr. Birgit Bertelsmeier \\
      \textbf{Zweiter Prüfer:} & <Name des 2. Prüfers> \\
    \end{tabular}
    \vspace{0.5cm}
    \\Gummersbach, im Januar 2022
  \end{center}
\end{titlepage}

%\newpage
%\begin{abstract}
%  Platz für das deutsche Abstract... %TODO
%\end{abstract}

%\renewcommand{\abstractname}{Abstract}
%\begin{abstract}
%  Platz für das englische Abstract...
%\end{abstract}
\newpage

% Inhaltsverzeichnis
\tableofcontents
\pagestyle{fancy} % Kopf- und Fußzeilen aktivieren (=> Paket "fancyhdr")

\newpage
\section*{Abbildungsverzeichnis}
\addcontentsline{toc}{section}{Abbildungsverzeichnis} % Manuellen Eintrag im Inhaltsverzeichnis erzeugen
\renewcommand{\listfigurename}{} % Name des Abbildungsverzeichnisses ändern
\thispagestyle{empty}
\listoffigures

%\newpage
%\clearpage
%\printglossaries
\section*{Glossar}%\label{Begriffe}
%%% MUSTER %%%
%\newglossaryentry{}
%{name=, description={}}

%%%%%%%%%%%%
%%% TIMO %%%
%%%%%%%%%%%%

\newglossaryentry{MVP}
{
    name=MVP,
    description={Minimum Viable Product, erste funktionsfähige Version des Produkts; dient dazu zu prüfen, ob es Sinn macht, das Produkt weiterzuentwickeln und ggf. anzupassen}
}


\newglossaryentry{Responsive Webdesign}
{
    name=Responsive Webdesign,
    description={Die Webseite wird so aufgebaut, dass sie auf möglichst vielen Endgeräten mit verschiedenen Bildschirmgrößen und Eingabemethoden (Maus, Touchscreen) skaliert wird und bedienbar ist.}
}

\newglossaryentry{DQL}
{
    name=DQL,
    description={Data Query Language, Datenabfragesprache für SQL}
}

\newglossaryentry{Tinder}
{
    name=Tinder,
    description={Datingportal, bei der Nutzer in Kontakt treten, indem sie sich bei der Kontaktsuche gegenseitig einen \enquote{like} geben}
}

\newglossaryentry{Brute-Force-Angriff}
{
    name=Brute-Force-Angriff,
    description={Versuch, ein Passwort zu knacken, indem jede mögliche Kombination nacheinander ausprobiert wird.}
}

\newglossaryentry{UI/UX}
{
    name=UI/UX,
    description={User Interface / User Experience. Benutzeroberfläche / Benutzerfreundlichkeit.} 
}

\newglossaryentry{MERN}
{
    name=MERN,
    description={Beliebter Techstack für das Web, bei dem MongoDB, ExpressJS, React und NodeJS verwendet werden}
}

\newglossaryentry{MongoDB}
{
    name=MongoDB,
    description={Beliebte Dokumentbasierte Datenbank}
}

\newglossaryentry{NodeJS}
{
    name=NodeJS,
    description={JS-Laufzeit, die es ermöglicht, JS auf dem Server auszuführen}
}

\newglossaryentry{Express}
{
    name=Express,
    description={Framework für NodeJS, welches die Entwicklung von Webanwendungen vereinfacht}
}

\newglossaryentry{React}
{
    name=React,
    description={JS UI-Framework für Webseiten}
}

\newglossaryentry{JavaScript}
{
    name=JavaScript,
    description={Beliebte Scriptsprache, die vor allem für Webseiten verwendet wird. Neben Web-Assembly die einzige Sprache, welche sich auf Klienten-Seite ausführen lässt. Nach ECMAScript standardisiert}
}

\newglossaryentry{REST-API}
{
    name=REST-API,
    description={REpresentional State Transfer; Architekturstil, der eine Schnittstelle über die Protokolloptionen POST (Create), GET (Read),  PUT/PATCH (Update) und DELETE (Delete) nach RFC-2616 ermöglicht}
}

\newglossaryentry{AWS S3}
{
    name=AWS S3,
    description={Amazon Simple Storage Service; Objektspeicher über Webschnittstelle}
}

\newglossaryentry{Multer}
{
    name=Multer,
    description={Middleware für NodeJS zum behandeln von POST-Anfragen des Typs \enquote{multipart/form-data}}
}

\newglossaryentry{Apollo}
{
    name=Apollo,
    description={Beliebte Implementation der GraphQl-Spezifikation}
}

\newglossaryentry{Mongoose}
{
    name=Mongoose,
    description={NodeJS-Paket zur Objektmodellierung von MongoDB}
}

\newglossaryentry{NoSQL}
{
    name=NoSQL,
    description={Not only SQL; Sammelbegriff für Datenbanken, die einen nicht-relationalen Ansatz verfolgen.}
}

\newglossaryentry{ORDBMS}
{
    name=,
    description={ObjektRelationales DatenBank ManagementSysten}
}

\newglossaryentry{ACID}
{
    name=ACID,
    description={Atomicity, Consistency, Isolation, Durability; häufig erwünschte Eigenschaften von Transaktionen von Datenbanken}
}

\newglossaryentry{JSON}
{
    name=JSON,
    description={JavaScipt Object Notation; Datenformat zum Austausch von Daten zwischen Anwendungen}
}

\newglossaryentry{Sharding}
{
    name=Sharding,
    description={Methode der Datenbankpartitionierung, Aufteilung der Datenbank in mehrere \enquote{Scherben}, die zusammen alle Daten verwalten}
}

\newglossaryentry{Replikatgruppe}
{
    name=Replikatgruppe,
    description={Gruppe von Datenbankprozessen, welche die selben Daten verwalten. Dadurch wird Redundanz und Hocherreichbarkeit gewahrt.}
}

\newglossaryentry{Constraint}
{
    name=Constraints,
    description={Bedingung, die eine Variable erfüllen muss, um in die Datenbank aufgenommen zu werden.}
}

\newglossaryentry{Multi-Master-Systeme}
{
    name=Multi-Master-Systeme,
    description={Datenbanksysteme, welche mehrere Prozesse ermöglichen, Schreibzugriffe durchzuführen.}
}

\newglossaryentry{Write-Ahead-Log}
{
    name=Write-Ahead-Log,
    description={Protokoll, welches vor jeder Datenänderung der PostgreSQL-Datenbank erstellt wird. Dies verringert die nötige Anzahl an Schribzugriffen und erleichtert das Erstellen und Aufspielen von Backups}
}

\newglossaryentry{Salt}
{
    name=,
    description={Wert, der dem zu hashenden Wert vor dem Hashen hinzugefügt wird, um die Wahrscheinlichkeit gleicher Eingabewerte zu verringern und damit die Informationssicherung zu erhöhen.}
}

\newglossaryentry{DDOS-Angriff}
{
    name=DDOS-Angriff,
    description={Distributed Denial Of Service Attack; Angriff eines Webdienstes durch Überlastung durch etliche Anfragen von einer Vielzahl von Geräten}
}

\newglossaryentry{DBaaS}
{
    name=DBaaS,
    description={DataBase as a Service; Datenbankinfrastruktur wird vom Anbieter aus der Cloud bezogen, statt diese selbst zu betreuen}
}

\newglossaryentry{Dev-Prod-Vergleichbarkeit}
{
    name=Dev-Prod-Vergleichbarkeit,
    description={Regel 10 der 12-Faktor-App, Entwicklungs- und Produktionsumgebung so ähnlich wie möglich halten, um die Probleme durch Inkompatibilitäten zu vermeiden}
}

\newglossaryentry{npm}
{
    name=npm,
    description={node package manager; Bibliothek und Paketmanager für Pakete der NodeJS-Runtime}
}

\newglossaryentry{RFC-7519}
{
    name=RFC-7519,
    description={Norm, auf der JSON Web Tokens basieren}
}

\newglossaryentry{JSON Web Token}
{
    name=JSON Web Token,
    description={Nach RFC-7519 genormter Zugriffsschlüssel, durch den die Identität des Nutzers einer Webseite gespeichert wird}
}

\newglossaryentry{API}
{
    name=API,
    description={Application Programming Interface; Programmierschnittstelle, durch die verschiedene Programme Daten austauschen können}
}

\newglossaryentry{Mutation}
{
    name=Mutation,
    description={Veränderung der bestehenden Daten nach GraphQl-Spezifikation, entspricht POST/PUT/PATCH/DELETE in REST-API}
}

\newglossaryentry{Angular}
{
    name=Angular,
    description={Angular ist ein Framework von Google für die Entwicklung von Mobil- und Webanwendungen. } 
}

\newglossaryentry{TypeScript}
{
    name=TypeScript,
    description={Superset von JS, wird zur Laufzeit nach JS transpiliert, bietet im Gegensatz zu JS starke Typisierung an, dadurch wird eine Quelle für Softwarefehler vermieden}
}

\newglossaryentry{ECMAScript 2015}
{
    name=ECMAScript 2015,
    description={ES2015, früher ES6; Eine Version von ECMAScript, die viele neue Funktionen bietet, JS basiert auf ECMAScript}
}

\newglossaryentry{XSS-Angriff}
{
    name=XSS-Angriff,
    description={Cross-Side-Scripting; Art der HTML-Injection, Kann auftreten, wenn Nutzereingaben mit Schadcode ungefiltert vom Server akzeptiert und ausgeführt werden}
}

\newglossaryentry{StackOverFlow}
{
    name=StackOverFlow,
    description={Beliebtes Programmierer-Forum}
}

\newglossaryentry{Github}
{
    name=Github,
    description={Beliebter Dienst zur Versionsverwaltung für Entwicklungsprojekte.}
}

\newglossaryentry{HTML-DOM}
{
    name=HTML-DOM,
    description={Das HTML-DOM ist ein Standardobjektmodell und eine Programmierschnittstelle für HTML.} 
}

\newglossaryentry{Jest}
{
    name=Jest,
    description={Beliebtes JS Testing-Framework. Stand 2022.}
}

\newglossaryentry{Cypress}
{
    name=Cypress,
    description={Cypress ist ein End-to-End-Test-Framework für die Automatisierung von Web-Tests.} 
}

\newglossaryentry{OpenBase}
{
    name=OpenBase,
    description={OpenBase hilft Entwicklern bei der Auswahl aus Millionen von Open-Source-Paketen} 
}

\newglossaryentry{Framework}
{
    name=Framework,
    description={Liefert einen Rahmen mit verschiedenen Funktionen, die für das Schreiben einer Anwendung verwendet und selektiv angepasst werden können.}
}

\newglossaryentry{atomare Tests}
{
    name=atomare Tests,
    description={Softwartests, welche in sich abgeschlossen sind und andere Tests nicht beeinflussen.}
}

\newglossaryentry{UTF-8}
{
    name=UTF-8,
    description={Weit verbreitete Kodierung für Unicode-Zeichen, sehr beliebt für Webseiten}
}

\newglossaryentry{DQL}
{
    name=DQL,
    description={Data Query Language, Datenabfragesprache für SQL}
}

%PRINTING BEGRIFFE
\textbf{ACID}\\
Atomicity, Consistency, Isolation, Durability; häufig erwünschte Eigenschaften von Transaktionen von Datenbanken.
\\\\
\textbf{atomare Tests}\\
Softwartests, welche in sich abgeschlossen sind und andere Tests nicht beeinflussen.
\\\\
\textbf{Angular}\\
Angular ist ein Framework von Google für die Entwicklung von Mobil- und Webanwendungen.
\\\\
\textbf{Apollo}\\
Beliebte Implementation der GraphQl-Spezifikation.
\\\\
\textbf{API}\\
Application Programming Interface; Programmierschnittstelle, durch die verschiedene Programme Daten austauschen können.
\\\\
\textbf{AWS S3}\\
Amazon Simple Storage Service; Objektspeicher über Webschnittstelle.
\\\\
\textbf{DQL}\\
Data Query Language, Datenabfragesprache für SQL
\\\\
\textbf{Destrukturierende Zuweisung}\\
Die destrukturierende Zuweisung ermöglicht es, Daten aus Arrays oder Objekten zu extrahieren, und zwar mit Hilfe einer Syntax, die der Konstruktion von Array- und Objekt-Literalen nachempfunden ist.
%\url{https//developer.mozilla.org/de/docs/Web/JavaScript/Reference/Operators/Destructuring_assignment}
\\\\
\textbf{MVP}\\
Minimum Viable Product, erste funktionsfähige Version des Produkts; dient dazu zu prüfen, ob es Sinn macht, das Produkt weiterzuentwickeln und ggf. anzupassen.
\\\\
\textbf{JSX}\\
Es heißt JSX und ist eine Syntaxerweiterung für JavaScript. JSX erinnert vielleicht an eine Template-Sprache, aber es verfügt über die volle Leistungsfähigkeit von JavaScript. JSX erzeugt React-"Elemente"
\\\\
\textbf{Over-Fetching}\\
Empfang von überschüssigen Daten durch eine Abfrage.
%Reference https//jwt.io/
\\\\
\textbf{undefined}\\
Eine Variable, der kein Wert zugewiesen wurde oder die überhaupt nicht deklariert wurde (nicht deklariert, existiert nicht), ist undefiniert. Eine Methode oder Anweisung gibt auch undefiniert zurück, wenn der ausgewerteten Variablen kein Wert zugewiesen wurde. Eine Funktion gibt undefiniert zurück, wenn kein Wert zurückgegeben wurde.
\\\\
\textbf{Responsive Webdesign}\\
Die Webseite wird so aufgebaut, dass sie auf möglichst vielen Endgeräten mit verschiedenen Bildschirmgrößen und Eingabemethoden (Maus, Touchscreen) skaliert wird und bedienbar ist.
\\\\
\textbf{UTF-8}\\
Weit verbreitete Kodierung für Unicode-Zeichen, sehr beliebt für Webseiten.
\\\\
\textbf{Web Token}\\
JSON-Web-Tokens sind eine dem Industriestandard RFC 7519 entsprechende Methode zur sicheren Darstellung von Forderungen zwischen zwei Parteien.
\\\\
\textbf{Tinder}\\
Datingportal, bei der Nutzer in Kontakt treten, indem sie sich bei der Kontaktsuche gegenseitig einen like geben.
\\\\
\textbf{Brute-Force-Angriff}\\
Versuch, ein Passwort zu knacken, indem jede mögliche Kombination nacheinander ausprobiert wird.
\\\\
\textbf{UI/UX}\\
User Interface / User Experience. Benutzeroberfläche / Benutzerfreundlichkeit.
\\\\
\textbf{MERN}\\
Beliebter Techstack für das Web, bei dem MongoDB, ExpressJS, React und NodeJS verwendet werden.
\\\\
\textbf{MongoDB}\\
Beliebte Dokumentbasierte Datenbank.
\\\\
\textbf{NodeJS}\\
JS-Laufzeit, die es ermöglicht, JS auf dem Server auszuführen.
\\\\
\textbf{Express}\\
Framework für NodeJS, welches die Entwicklung von Webanwendungen vereinfacht.
\\\\
\textbf{React}\\
JS UI-Framework für Webseiten.
\\\\
\textbf{JavaScript}\\
Beliebte Scriptsprache, die vor allem für Webseiten verwendet wird. Neben Web-Assembly die einzige Sprache, welche sich auf Klienten-Seite ausführen lässt. Nach ECMAScript standardisiert.
\\\\
\textbf{REST-API}\\
REpresentional State Transfer; Architekturstil, der eine Schnittstelle über die Protokolloptionen POST (Create), GET (Read),  PUT/PATCH (Update) und DELETE (Delete) nach RFC-2616 ermöglicht.
\\\\
\textbf{Multer}\\
Middleware für NodeJS zum behandeln von POST-Anfragen des Typs multipart/form-data.
\\\\
\textbf{Mongoose}\\
NodeJS-Paket zur Objektmodellierung von MongoDB.
\\\\
\textbf{NoSQL}\\
Not only SQL; Sammelbegriff für Datenbanken, die einen nicht-relationalen Ansatz verfolgen.
\\\\
\textbf{ORDBMS}\\
ObjektRelationales DatenBank ManagementSysten.
\\\\
\textbf{JSON}\\
JavaScipt Object Notation; Datenformat zum Austausch von Daten zwischen Anwendungen.
\\\\
\textbf{Sharding}\\
Methode der Datenbankpartitionierung, Aufteilung der Datenbank in mehrere Scherben die zusammen alle Daten verwalten.
\\\\
\textbf{Replikatgruppe}\\
Gruppe von Datenbankprozessen, welche die selben Daten verwalten. Dadurch wird Redundanz und Hocherreichbarkeit gewahrt.
\\\\
\textbf{Constraint}\\
Bedingung, die eine Variable erfüllen muss, um in die Datenbank aufgenommen zu werden.
\\\\
\textbf{Multi-Master-Systeme}\\
Datenbanksysteme, welche mehrere Prozesse ermöglichen, Schreibzugriffe durchzuführen.
\\\\
\textbf{Write-Ahead-Log}\\
Protokoll, welches vor jeder Datenänderung der PostgreSQL-Datenbank erstellt wird. Dies verringert die nötige Anzahl an Schribzugriffen und erleichtert das Erstellen und Aufspielen von Backups.
\\\\
\textbf{Salt}\\
Wert, der dem zu hashenden Wert vor dem Hashen hinzugefügt wird, um die Wahrscheinlichkeit gleicher Eingabewerte zu verringern und damit die Informationssicherung zu erhöhen.
\\\\
\textbf{DDOS-Angriff}\\
Distributed Denial Of Service Attack; Angriff eines Webdienstes durch Überlastung durch etliche Anfragen von einer Vielzahl von Geräten.
\\\\
\textbf{DBaaS}\\
DataBase as a Service; Datenbankinfrastruktur wird vom Anbieter aus der Cloud bezogen, statt diese selbst zu betreuen.
\\\\
\textbf{Dev-Prod-Vergleichbarkeit}\\
Regel 10 der 12-Faktor-App, Entwicklungs- und Produktionsumgebung so ähnlich wie möglich halten, um die Probleme durch Inkompatibilitäten zu vermeiden.
\\\\
\textbf{npm}\\
node package manager; Bibliothek und Paketmanager für Pakete der NodeJS-Runtime.
\\\\
\textbf{RFC-7519}\\
Norm, auf der JSON Web Tokens basieren.
\\\\
\textbf{JSON Web Token}\\
Nach RFC-7519 genormter Zugriffsschlüssel, durch den die Identität des Nutzers einer Webseite gespeichert wird.
\\\\
\textbf{Mutation}\\
Veränderung der bestehenden Daten nach GraphQl-Spezifikation, entspricht POST/PUT/PATCH/DELETE in REST-API.
\\\\
\textbf{TypeScript}\\
Superset von JS, wird zur Laufzeit nach JS transpiliert, bietet im Gegensatz zu JS starke Typisierung an, dadurch wird eine Quelle für Softwarefehler vermieden.
\\\\
\textbf{ECMAScript 2015}\\
ES2015, früher ES6; Eine Version von ECMAScript, die viele neue Funktionen bietet, JS basiert auf ECMAScript.
\\\\
\textbf{XSS-Angriff}\\
Cross-Side-Scripting; Art der HTML-Injection, Kann auftreten, wenn Nutzereingaben mit Schadcode ungefiltert vom Server akzeptiert und ausgeführt werden.
\\\\
\textbf{StackOverFlow}\\
Beliebtes Programmierer-Forum.
\\\\
\textbf{Github}\\
Beliebter Dienst zur Versionsverwaltung für Entwicklungsprojekte.
\\\\
\textbf{HTML-DOM}\\
Das HTML-DOM ist ein Standardobjektmodell und eine Programmierschnittstelle für HTML.
\\\\
\textbf{Jest}\\
Beliebtes JS Testing-Framework. Stand 2022.
\\\\
\textbf{Cypress}\\
Cypress ist ein End-to-End-Test-Framework für die Automatisierung von Web-Tests.
\\\\
\textbf{OpenBase}\\
OpenBase hilft Entwicklern bei der Auswahl aus Millionen von Open-Source-Paketen.
\\\\
\textbf{Framework}\\
Liefert einen Rahmen mit verschiedenen Funktionen, die für das Schreiben einer Anwendung verwendet und selektiv angepasst werden können.
\\\\

\section*{Akronyme:}
{AWS}: {Amazon Web Services}\\
{DOM}: {Document Object Model}\\
{API}: {Application Programming Interface}\\
{AJAX}: {Asynchronous JavaScript And XML}\\
\addcontentsline{toc}{section}{Glossar} 

\newpage
\section*{Struktur der Arbeit}\label{kap_struktur}
\addcontentsline{toc}{section}{Struktur der Arbeit} % Manuellen Eintrag im Inhaltsverzeichnis erzeugen
%\subsection*{Einführung in das Thema }%(Motivation, zentrale Begriffe etc)
%\addcontentsline{toc}{subsection}{Einführung in das Thema}
%\subsection*{Hinführung zu den Ergebnissen}
%\addcontentsline{toc}{subsection}{Hinführung zu den Ergebnissen} 
%\subsection*{Ggf. Angabe des Schwerpunktes}
%\addcontentsline{toc}{subsection}{Angabe des Schwerpunktes} 
%\subsection*{Ggf. Einschränkungen darlegen}
%\addcontentsline{toc}{subsection}{Einschränkungen darlegen} 
%\subsection*{Problemstellung}
%\addcontentsline{toc}{subsection}{Problemstellung} 
%\subsection*{Zielstellung der Arbeit}
%\addcontentsline{toc}{subsection}{Zielstellung der Arbeit} 
%\subsection*{Fragestellung der Arbeit}
%\addcontentsline{toc}{subsection}{Fragestellung der Arbeit} 
%\subsubsection*{Struktur der Arbeit}
%\addcontentsline{toc}{subsection}{Struktur der Arbeit} 
Dieser Projektbericht ist in folgende Kapitel unterteilt:\\
In \textbf{Kapitel \ref{kap_anforderungen}} wird das Projekt in allgemeiner Form sowie seine Anforderungen beschrieben.
\\\\
In \textbf{Kapitel \ref{kap_Technologieinfrastruktur}} wird kurz auf die unterschlichen Technologien und die Architektur der Applikation eingegangen.
\\\\
In \textbf{Kapitel \ref{kap_Datenbank}} werden die Datenbanken PostgreSQL und MongoDB gegenübergestellt. Außerdem wird das Datenbankschema erläutert.
\\\\
Das \textbf{Kapitel \ref{kap_Backend}} beschäftigt sich mit den Elementen, die die Verbindung zwischen dem Frontend und der Datenbank ermöglichen. Außerdem wird erklärt, wie die Authentifizierung und Autorisierung mit Hilfe des JWT funktioniert.
\\\\
 \textbf{Kapitel \ref{kap_Schnittstelle}} wird näher auf die Abfragen eingehen, die von dem Frontend an die Datenbanken mittels GraphQL erfolgen.
\\\\
%Anhand von zwei Beispielen wird gezeigt, wie Lese- und Schreibabfragen mit Hilfe von GraphQL und ApolloClient durchgeführt wurden.
In \textbf{Kapitel \ref{kap_Frontend}} wird erläutert, welche JavaScript-Frameworks zu Beginn des Projektes berücksichtigt wurden. Schließlich wird gezeigt, wie die Daten für den Endbenutzer visualisiert werden.
\\\\
%Qualitätssicherung
Die Relevanz von Testfällen in Softwareprojekten wird in \textbf{Kapitel \ref{kap_QS}} dargestellt. Es werden ebenfalls die Vorteile automatisierter und eingebetteter Testfälle in der Entwicklungsumgebung besprochen.



\newpage
\section{Anforderungen}\label{kap_anforderungen}
Im Rahmen des Praxisprojekts wird ein Minimum Viable Product
(\gls{MVP}) 
für eine Anwendung entwickelt, auf der sich Nutzer kennenlernen und miteinander interagieren können. Zielgruppe der Anwendung sind Spieler des Videospiels \textit{League of Legends}, welche nach neuen Mitspielern für gemeinsame Partien suchen.

Um die Anwendung im vollen Umfang nutzen zu können, muss sich der Nutzer ein Konto erstellen.
Dazu sind Angaben über Email-Addresse und gewünschtem Nutzernamen und Passwort notwendig.
Nach der Registrierung kann sich der Nutzer auf dem eigenen Profil beschreiben.
Denkbare Felder sind ein Freitext, ein Profilbild, das Alter, das Geschlecht, präferierte Spielpositionen und Helden sowie Lieblingsspielmodi.
Die Selbstbeschreibung soll anderen Nutzern helfen, einschätzen zu können, ob man zueinander passt.\\

%Suche
Andere Nutzer sollen über eine Suchfunktion findbar sein.
Neben einer Suche ohne Parameter, welche zufällige Nutzer anzeigt, soll eine Filterfunktion erstellt werden.
Diese erlaubt es, die Suche auf Nutzer mit den gewählten Eigenschaften, zum Beispiel einem bestimmten Geschlecht oder einer präferiterten Spielposition, einzugrenzen. \\

%Freunde
Das Befreunden mit anderen Nutzern soll ähnlich wie auf dem Datingportal \textit{Tinder} gelöst werden: Nutzern kann ein \textit{Like} (\enquote{mag ich}) gesendet werden.
Dies sorgt dafür, dass die suchende Person in der nächsten Suche der gesuchten Person oben angezeigt wird.
Wenn die gesuchte Person den \textit{Like} erwiedert, werden die beiden Nutzer befreundet.
Befreundete Nutzer sollen in der Lage sein, miteinander über einen Chat zu kommunizieren.
Dieser Chat erlaubt das Schreiben von Text nach \gls{UTF-8} Spezifikation,
dies ermöglicht das schicken von Emojis.\\

Fehlverhalten, wie zum Beispiel die Bewerbung von kostenpflichtigen Diensten sowie explizite Inhalte und aufdringliches Verhalten sind unerwünscht.
Nutzer sollen daher in der Lage sein, solches Verhalten zu melden.
Meldungen werden durch Moderatoren verfolgt und ggf. bestraft.
Auch soll das Blockieren von anderen Nutzern möglich sein.
Blockierte Nutzer sollen automatisch von der Freundesliste entfernt werden und werden in der Suche nicht mehr angezeigt.
Wenn ein Nutzer Fehlverhalten meldet, wird der gemeldete Nutzer automatisch blockiert.\\

Sollte ein Nutzer mit dem Produkt nicht zufrieden sein, steht es ihm frei, das Konto zu löschen.
Hier bietet es sich an, den ehemaligen Nutzer nach Verbesserungsvorschlägen zu fragen, um Makel ausbessern zu können.\\

Die Nutzer sollen in der Lage sein, sich ein Benutzerkonto erstellen, auf dem Profil Angaben über die eigene Person machen, andere Nutzer per Suchfunktion finden und Freundschaftsanfragen schicken zu können.
Personen, die miteinander befreundet sind, können im Chat Nachrichten austauschen.
Dadurch können zum Beispiel Termine zum gemeinsamen Spielen vereinbart, Kontaktdaten ausgetauscht und Freundschaften geschlossen werden.\\

%Auf der Startseite soll ein Interessent einen ersten Eindruck von der Anwendung erhalten, bevor dieser sich ein Konto erstellen muss.

Um viele Nutzer erreichen zu können, soll die Anwendung in Englisch verfügbar sein.
Durch  \gls{Responsive Webdesign}
soll die Anwendung Desktop und Mobil unterstützen können, dies erhöht die potenzielle Nutzerbasis.\\

Aus Softwaresicht sollte auf eine gute Dokumentation geachtet werden, um die Wartbarkeit des Projektes zu erhöhen und Kollaboratoren einen einfacheren Einstieg zu ermöglichen.
Um die Chance von erfolgreichen Brute-Force-Angriffen auf Passwörter zu verringern, sollen sinnvolle Anforderungen an das vom Nutzer gewählte Passwort gestellt werden.
Auch sollen die Passwörter ausreichend verschlüsselt werden, um die Auswirkungen von Datenbankangriffen zu verringern.\\

Da es sich bei dem Projekt um ein MVP handelt, sollen nur die nötigsten Grundfunktionen entwickelt werden.
Themen wie UI/UX und Finanzierung liegen daher nicht im Rahmen des Projekts.
Das Ziel ist, die Idee einer Kennenlern-Anwendung zu testen und zu prüfen, ob es sich lohnt, das Projekt weiterzuentwickeln. 

\newpage
\section{Technologieinfrastruktur}\label{kap_Technologieinfrastruktur}
\paragraph{}
Um den Nutzern eine ansprechende Benutzeroberfläche zu bieten, durch die der Klient über eine Schnittstelle mit dem Server kommunizieren und schlussendlich unsere Webseite bedienen kann, ist eine Mehrzahl an ineinandergreifenden Technologien (eng.: Techstack) notwendig.

\paragraph{}
Es wurde sich für MERN entschieden, einem ausgereiften Techstack bestehend aus MongoDB als Datenbank, NodeJS und Express im Backend und React im Frontend.
Diese Technologien eignen sich für die Entwicklung von Webapplikationen mit hoher Qualität und erlaubt eine schnelle Entwicklung \cite{ti:mernStackAdvantages}.
Alle genannten Technologien verwenden die Programmiersprache JavaScript.
Dies erleichtert das Programmieren, da nicht zwischen den Syntaxen verschiedener Sprachen gewechselt werden muss.
Auch kann der gleiche Programmiercode an verschiedenen Stellen des Projektes wiederverwendet werden, was Zeit und Kosten spart.
Die Technologien sind ausgereift und werden vielfach aktiv von führenden Technologiegiganten verwendet \cite{ti:mongo}\cite{ti:express}\cite{ti:react}\cite{ti:node}.
Außerdem wird für die Programmierschnittstelle statt einer typischen REST-API GraphQL (Graph Query Language) verwendet. \\

\includegraphics[width=\textwidth*0.95]{sources/MERN-Stack_Schaubild.drawio}
\begin{figure}[ht]
	\centering
	\caption{Die Interaktion der verwendeten Technologien als Schaubild}
	\label{figMERN1}
\end{figure}

\paragraph{}
Der Nutzer interagiert mit dessem Endgerät mit der durch React generierten Benutzeroberfläche.
Für dynamische Inhalte werden dazu Anfragen an das NodeJS-Backend gesendet.
Dieses verwaltet mithilfe von Express verschiedene Routen.
Bilder werden als Dateien über die Route \textit{/api} mit Multer auf AWS S3 gespeichert.
Sollten Daten der Datenbank benötigt werden, wird die Route \textit{/graphql} verwendet, bei denen über Apollo und Mongoose die entsprechende Abfrage an die MongoDB-Datenbank geschickt wird. 
Die Technologien sowie dessen Alternativen und die Gründe zur Entscheidung werden in den folgenden Kapiteln erläutert.


\newpage
\section{Datenbank}\label{kap_Datenbank}
% Welche Datenbank wurde verwendet und warum,...
\subsection{Anforderungen}
Es wird eine Plattform entwickelt, auf der sich Nutzer ähnlich Social Media Profile anderer Nutzer anschauen und Chats miteinander führen. Nutzer sollen in Echtzeit miteinander schreiben können und nahezu keine Wartezeit in Kauf nehmen müssen, um sich andere Profile anzeigen zu lassen. Die Datenbank muss nahezu immer erreichbar sein, da Nutzer unsere Webseite sonst nicht nutzen können. 
Wenn Benutzer Änderungen an ihrem Profil vornehmen, kann eine geringe Wartezeit in Kauf genommen werden. Nutzer müssen bei Profilen nicht unbedingt die aktuelleste, gerade veränderte Profilbeschreibung sehen; es ist viel wichtiger, dass die Nutzer Nutzerprofile schnell sehen, all dass diese auf dem neuesten Stand sind. 
Webseiten wie unsere können "über Nacht" zum Erfolg werden, es reicht ein großer Influencer, welcher seinen Followern von der Webseite berichtet und auf einen Schlag erreichen wir Nutzerzahlen, bei denen unsere Rechenleistung auf Grenzen stößt. Die Datenbank unserer Wahl sollte sich vor allem schnell, ohne viel Programmieraufwand, skalieren lassen, um die Zeit, in denen unsere Plattform ausgelastet ist, möglichst gering zu halten. 
Statt dass unsere Nutzer ihr Verhalten auf unsere Webseite anpassen müssen, wollen wir die Webseite auf die Wünsche unserer Nutzer anpassen. Datenstrukturen sollen sich schnell und oft ändern können, um dies zu ermöglichen.

\subsection{Wahl der Datenbank}
In der näheren Auswahl der Datenbank standen die SQL-Datenbank PostgreSQL und MongoDB, eine NoSQL-Datenbank, zwei beliebte Datenbankmanagementsysteme mit unterschiedlichen Ansätzen. Wir werden uns im Folgenden die Gemeinsamkeiten und Unterschiede der Datenbanken anschauen und erklären, warum wir MongoDB als die für unser Projekt beste Lösung halten.

\subsubsection{PostgreSQL}
Michael Stonebreaker veröffentlichte 1974 in Berkeley unter BSD-Lizenz das RDBMS INGRES. Jahre später führte er das Projekt als Postgres (Post INGRES) weiter und fügte einen objektorientierten Ansatz hinzu. \cite{PG1} Im Laufe der Jahre wurde PostgreSQL fortlaufend weiterentwickelt und hat viele neue Funktionen dazuerhalten.
PostgreSQL beschreibt sich selbst als das fortschrittlichste, quelloffene, relationale Datenbankmanagementsystem und habe sich in über 30 Jahren aktiver Entwicklung einen guten Ruf für Zuverlässlichkeit, Funktionsrobusheit und Leistung verdient. \cite{PG2} PostgreSQL wird auch in Zukunft unter einer quelloffenen Lizenz stehen, welche die kommerzielle Nutzung erlaubt. \cite{PG3}
PostgreSQL wurde exemplarisch als Option eines SQL-Systems verwendet, da der Funktionsumfang vergleichbar mit anderen beliebten SQL-Systemen ist. \cite{PG4}

\subsubsection{MongoDB und PostgreSQL im Vergleich}

\subsubsection{MongoDB}
\begin{quote}
Die 2007 neu gegründete Firma 10Gen, mittlerweile bekannt als MongoDB Inc., benötigte eine Datenbank, welche den Anforderungen ihrer quelloffenen Plattform-as-a-Service Cloud-Architektur gerecht werden würde. Das Team suchte nach einer Datenbank, die elastisch, skalierbar, einfach zu verwalten und für Entwickler und Anwender einfach zu benutzen ist. Unzufrieden mit den auf dem Markt verfügbaren Datenbanksystemen wurde MongoDB, eine dokumentbasierte Datenbank entwickelt. Als das Team das Potenzial der Datenbank realisierte, wurde die Idee der Cloud-Plattform eingestellt und die Entwicklung von MongoDB gefördert.\cite{MG1}
Laut eigenen Aussagen ist 
\end{quote}
MongoDB [] eine universelle, dokumentbasierte, verteilte Datenbank für die moderne Anwendungsentwicklung und die Cloud, die in puncto Produktivität höchsten Ansprüchen gerecht wird. \cite{MG2} 


\subsubsection{MongoDB und PostgreSQL im Vergleich}

\begin{center}
    \begin{tabular}{ |c|c|c| } 
     \hline
     Kriterien & PostgreSQL & MongoDB  \\ 
     \hline
     Datenbanktyp & SQL, Objektrelational & NoSQL, Dokumentbasiert \\
     Ranking nach DB-Engines \cite{DB1} & 577,50 Punkte, viertbeliebteste Datenbank, viertbeliebteste relationale Datenbank & 496,50 Punkte, fünftbeliebteste Datenbank, beliebteste nicht-relationale Datenbank \cite{DB2} \\
     Architektur \\ Monolithisch \\ Dezentralisiert
     Erscheinungsjahr & 1989 & 2009 \\
     Datenschema & Ja & Selbstbeschreibende Dokumente\\
     Referenzierung & Referenzierung per ID, erlaubt Foreign Keys & Referenzierung per ID oder eingebettetes Sub-Dokument \\
     Datenstruktur & Tabellen bestehen aus Zeilen und Spalten & Kollektionen bestehen aus Dokumenten. Dokumente haben Felder. Dokumente der gleichen Kollektion müssen nicht die selben Felder besitzen. \\
     Typisierung & Erlaubt verschiedene Datentypen und nutzerdefinierte Datentypen\cite{PG5} & Erlaubt verschiedene Datentypen nach BSON Spezifikation \cite{MG3}; ähnelt stark JSON \\
     Horizontale Skalierung & Repliken nach Master-Slave-System \cite{PG6} & Sharding und Replica Sets als Teil der Infrastruktur \cite{MG4} \cite{MG5} \\
     Konsistenzmodell & ACID, optional eventuelle Konsistenz \cite{MG6} \cite{MG7}  & ACID \\
     CAP-Theorem & CA & CP/AP \\ %Correct?
     \hline
    \end{tabular}
    \cite{DB3} \cite{DB4}
\end{center}

\subsubsection{ACID}
Auch wenn man NoSQL oft mit BASE verbindet, muss eine NoSQL Datenbank nicht zwangsläufig den Prinzipien von BASE folgen. MongoDB ist seit Erschaffung ACID-Konform auf Dokumentebene und unterstützt ab den Versionen 4.0 (Juni 2018 \cite{MG8}) auf einem einzelnen Replik-Set bzw. ab Version 4.2 (August 2019 \cite{MG8}) zwischen mehreren Sets multi-Dokument-Transaktionen mit ACID-Konformität. \cite{MG6} Dies erlaubt "Alles-oder-Nichts"-Transaktionen, bei denen es kritisch ist, dass entweder alle Teile oder kein Teil der Transaktion ausgeführt wird und löst damit Probleme, die historisch nur mit SQL-Datenbanken lösbar waren.

\subsubsection{Referenzierung und Denormalisierung}
PostgreSQL erlaubt es, andere Tabellen mit Foreign Keys (FK) gemäß SQL-Standard zu referenzieren. Der Wert eines FK muss korrekt sein, sollte der angegebene FK in der referenzierten Tabelle nicht existieren, wird ein Fehler geworfen. Auch ist es möglich, entsprechende Regeln zu definieren, was passieren soll, wenn der referenzierte Datensatz beispielsweise gelöscht wird.
MongoDB erlaubt auch, andere Dokumente per ID zu referenzieren, bietet jedoch keine direkte Option, direkt zu definieren, was beim löschen des referenzieren Datensatz passieren soll. Es ist weiterhin möglich, Trigger beim Löschen des referenzierten Datensatzes feuern zu lassen, dies ist im Vergleich zum Foreign Key Constraint jedoch aufwändig. Stattdessen setzt MongoDB darauf, alle relevanten Informationen in einem Dokument einzubetten, soweit möglich.
%TODO Schaubilder erstellen 1-1 1-Many Many-Many

Denormalisierung und damit eingebettete Dokumente erlauben schnelleren Lesezugriff, da alle relevanten Informationen in einem Dokument sind und anders als in SQL nicht über mehrere Tabellen mit über Zeit immer länger werdenden JOIN-Anweisungen zusammengefasst werden müssen. Dies erleichtert Abfragen und erhöht die Geschwindigkeit von Leseabfragen. Der Nachteil ist, dass Daten über verschiedene Dokumente dupliziert werden und Schreibzugriffe langsamer werden - schließlich müssen Daten in verschiedenen Dokumenten erstellt oder aktualisiert werden. 
Wir glauben, dass viele Nutzer ihre Profilinformationen wie Avatarbild, Name oder Profiltext selten verändern. Jedes Mal, wenn das Profil eines Nutzers in der Kontaktsuche gezeigt werden soll, jedes Mal, wenn die Freundesliste oder ein Chat geöffnet wird, muss das Profil abgefragt werden - teilweise hunderte Male am Tag. Eine schnelle Antwort ist hier auch wichtig: Der Nutzer will nicht warten müssen, bis die Seite geladen hat. Auf der anderen Seite wird ein Nutzer vermutlich selten seine Profilinformationen ändern, höchstens ein paar Mal am Tag. Wir glauben auch, dass der Nutzer hier eher geneigt ist, eine kurze Verzögerung für die Aktualisierung seiner Daten hinzunehmen. Entsprechend sind wir der Meinung, dass
die Vorteile der Denormalisierung dessen Nachteile überwiegen.

\subsubsection{Skalierbarkeit}
Sowohl PostgreSQL als auch MongoDB lassen sich vertikal skalieren - mit mehr Ressourcen läuft die Datenbank schneller. Bei horizontaler Skalierung verfolgen die beiden Datenbanken verschiedene Ansätze.
\paragraph{PostgreSQL} nutzt ein Master-Slave-System bzw. Primary-Standby-System mit Load Balancing.\cite{PG6} Die Standby-Knoten sind Kopien des Primärknotens und können Lesezugriffe verarbeiten. Schreibzugriffe werden nur vom Primärknoten angenommen. Sollte der Primärknoten versagen, bietet PostgreSQL keine automatische Lösung dieses Problems, ohne Drittsoftware muss manuell ein neuer Primärknoten gewählt werden. Es gibt nur einen Primärknoten, für Multi-Master-Systeme wird Drittsoftware benötigt. Selbst mit synchronen Repliken dauert es einen Moment, bis die Standby-Knoten die Aktualisierungen der Datenbank übernommen haben, dies kann dazu führen, dass Abfragen mit veralteten Datensätzen beantwortet werden.\cite{PG8} Consistency nach CAP-Theorem ist für PostgreSQLs Master-Slave-Systeme somit nicht mehr einhundertprozentig gegeben.
Durch Sharding lässt sich die Datenbank in einzelne Knoten aufteilen, die jeweils nur einen Teil der Daten beinhalten. Dies erlaubt es, die Last und benötigte Speicherkapazität pro Knoten weiter zu verringern. Es ist möglich, mehrere Shards vom gleichen Knoten verwalten zu lassen; dies erleichtert die Skalierung, da die Anzahl der Shards flexibel an die technischen Ressourcen des verwaltenden Knoten angepasst werden kann.
\paragraph{MongoDB}
benutzt Replica-Sets, welche ähnlich wie das vorgestellte Master-Slave-System von PostgreSQL funktionieren. Der Primärknoten ist der einzige Knoten mit Schreibzugriff und repliziert die Änderungen auf die Sekundärknoten. Alleridngs hat MongoDB ein automatisches System, welches einen Ausfall des Primärknotens abfängt. Alle Knoten teilen den anderen Knoten ihren "Herzschlag" mit, pingen sich gegenseitig an. Sollte der Herzschlag des Primärknotens ausfallen, wählen die Sekundärknoten unter ihnen einen neuen Primärknoten aus, welcher dann die Aufgaben des früheren Primärknotens übernimmt. Es ist möglich, Knoten mit mehr Rechenleistung eine höhere Priorität zuzuweisen, um die Wahrscheinlichkeit zu erhöhen, dass dieser Knoten der nächste Primärknoten wird. Auch ist es möglich, zu verhindern, dass spezielle Knoten Primärknoten werden, dies ist ratsam für Knoten, die schlechtere Hardware besitzen oder eine höhere Latenz aufweisen. Den Knoten können zudem verschiedene Rollen zugewiesen werden, so können versteckte Knoten für Datenbankauswertungen verwendet werden, während verzögerte Knoten einen historischen Schnappschuss der Datenbank speichern und Aktualisierungen mit einer Verzögerung ausführen und somit als Backup dienen können. \cite{MG10}\cite{MG11}\cite{MG12}\cite{MG13} MongoDB erlaubt auch Systeme mit mehreren Primärknoten, es ist jedoch empfohlen, in den meisten Fällen Optionen wie Sharding zu verwenden. \cite{MG14}
Shards wiederum können als eigene Replica-Sets eingesetzt werden. Dies erlaubt es, für verschiedene Regionen eigene Shards der Datenbank einzurichten, welche dann wiederum in einem eigenen Replica-Set gesteuert werden.
%TODO schaubild
Im Laufe der jahre und mit immer größeren Datenmengen ist die Möglichkeit, horizontal skalieren zu können immer wichtiger geworden. Zur Anfangszeit von PostgreSQL hat es in den meisten Fällen gereicht, vertikal zu skalieren, erst über die Jahre wurden Techniken zur horizontalen Skalierung entwickelt. Für Techniken wie der automatischen Wahl eines neuen Primärknotens oder Multi-Master-Systemen benötigt es Drittsoftware. Horizontale Skalierung war für MongoDB schon immer ein wichtiges Thema und ist als solches stark in die Datenbankinfrastruktur eingebettet. Wir sind der Meinung, dass die von MongoDB verwendeten Lösungen zur horizontalen Skalierung ausgereifter sind und gleichzeitig weniger fachliches Wissen benötigen, damit also schneller durchfürbar sind.

\subsubsection{Flexibilität}
Gerade in der Anfangsphase von Projekten werden Datenbankstrukturen oft umgeworfen, angepasst und verworfen. Dokumente einer Kollection müssen, anders als Spalten einer ORDBMS-Tabelle, nicht die gleiche Struktur aufweisen. Sollten also Änderungen an einem Dokuemtentyp vorgenommen werden, müssen vorherige Daten nicht angepasst und bereinigt werden. Dies erleichtert einen agilen Programmieransatz, kann aber die technischen Schulden des Projektes erhöhen. Wir glauben, dass es im Laufe des Lebenszyklus unseres Projektes oft zu kleineren Anpassungen in Datenbankstrukturen kommen wird und die Flexibilität von MongoDB im Vergleich zu PostgreSQL den Programmieraufwand verringern wird.

\subsubsection{Das CAP-Theorem}
%TODO gut schreiben
Brewers CAP-Theorem sagt aus, dass es in verteilten Systemen wie Datenbanken nicht möglich ist, gleichzeitig Consistency (Konsistenz), Availability (Verfügbarkeit) und Partition Tolerance (Partitionstoleranz) zu gewährleisten. Man muss sich für zwei entscheiden. %cite paper
% https://awoc.wolski.fi/dlib/big-data/GiLy02-CAP.pdf AXIOM
% https://groups.csail.mit.edu/tds/papers/Gilbert/Brewer2.pdf
% https://www.researchgate.net/publication/220476881_CAP_Twelve_years_later_How_the_Rules_have_Changed (download CAP-computer)
%TODO weiter recherchieren https://de.wikipedia.org/wiki/CAP-Theorem
Dies ist eine starke Simplifizierung, die den heutigen Datenbanken in der Form nicht gerecht wird und zwölf Jahre später von Brewer angepasst wurde. 
Viele aktuelle Datenbanken erlauben es, in unterschiedlichen Konfigurationen verschiedene Ziele zu erreichen. 
  
Wir haben am Fallbeispiel von PostgreSQL zeigen können, dass im Multiknoten-System ein Teil der Konsistenz aufgegeben werden kann, um kürzere Antwortzeiten und Partitionstoleranz (Standby-Knoten dürfen ausfallen) zu gewährleisten. Die Einführung eines Multi-Master-Systems würde eine Erhöhung der Partitionstoleranz (beliebige Knoten dürfen ausfallen) zu Kosten von Konsistenz (gleichzeitige Schreibzugriffe auf verschiedene Knoten) erwirken und somit das System Richtung AP-Datenbank verschieben. Synchrone Replizierung auf Standby-Knoten erhöht die Konsistenz, erhöht aber gleichzeitig die benötigte Zeit einer Operation und verringert somit die Verfügbarkeit, schiebt also somit den Fokus Richtung CP-Datenbank. Diese Funktionen sind in anderen Datenbanken in ähnlicherweise zu finden und erlauben aktuellen Datenbanken, fein auf die Anforderungen des Benutzers angepasst zu werden.


\subsubsection{Dateiformat}
MongoDB speichert Daten im \glqq binary JSON\grqq -Format. JSON wiederum steht für \glqq JavaScript Object Notation \grqq und 
\begin{quote}
    ist ein schlankes Datenaustauschformat, welches für Menschen einfach zu lesen und für Maschinen einfach zu parsen [] ist
\end{quote} \cite{JSON1}. 
JSON als semistrukturiertes Dateiformat eignet sich gut für Schnittstellendaten. Zudem wird im gewählten MERN-Techstack ausschließlich JavaScript verwendet - das JavaScript native Dateiformat JSON ist daher ohne Umwandlungen direkt verwendbar und der Umgang für unsere Entwickler bereits bekannt. Dies verringert die Gefahr möglicher Komplikationen und spart Lern- und Programmieraufwand.

\subsubsection{Beliebtheit}
Eine Datenbank zu wählen, die beliebt ist, hat einige Vorteile. Beliebte Datenbanken werden meist aktiv weiterentwickelt, haben mehr Tools von Drittanbietern, die die Arbeit erleichern und haben eine aktive Programmierergemeinschaft, welche einem bei kleineren Problemen schnell aushelfen kann. Sollten wir uns entscheiden zu expandieren, ist es zudem einfacher, qualifiziertes Fachpersonal für die entsprechenden Datenbanken zu finden.
Nach dem Ranking von DB-Engines schlagen PostgreSQL und MongoDB ähnlich ab... <Stats raussuchen>
% https://insights.stackoverflow.com/survey/2020#technology-databases-all-respondents4
Beide Datenbanken erfreuen sich großer Beliebtheit und werden allen Anschein nach noch etliche Jahre verwendet werden. Im NoSQL-Markt ist MongoDB am beliebtesten, während es im SQL-Markt weitere andere beliebte Datenbanken gibt. Beide Datenbanken sind beliebt genug, dass in diesem Aspekt keine Datenbank Vorteile gegenüber der anderen hat.

\subsubsection{Erfahrung}
Das Entwicklerteam hat in der Vergangenheit bereits Erfahrung in MongoDB sammeln können und ist gut mit der Datenbank zurecht gekommen. Die Entwickler wissen, wie die Datenbank funktioniert und auf was zu achten ist. Dies Verringert das Risiko und spart Zeit, da die Technologie bereits bekannt ist.

\subsubsection{Database-as-a-Service}
Uns Fehlen die Kapazitäten und die Infrastruktur, um selbst die Datenbank zu betreiben. Stattdessen sind wir auf eine DataBase-as-a-Service-Lösung angewiesen. Diese lassen sich schnell einrichten, fallen selten aus und werden automatisch mit Updates versorgt. Es stehen Reportingtools zur Verfügung, die das Auswerten der Datenbank erleichern. DBaaS spart viel administrativen Aufwand und verringert damit die Kosten.
MongoDB Inc. bietet mit MongoDB Atlas eine Database-as-a-Service Lösung an, die Flexibel auf die Größe und Auslastung des Projektes angepasst werden kann. Dazu gibt es verschiedene Datenbankstufen, die mit höheren Kosten mehr Rechenleistung und weitere Funktionen erhält. Zwischen den Stufen kann flexibel gewechselt werden, um den aktuellen Anforderungen gerecht zu werden. Kostenpflichtige Stufen bieten die Möglichkeit von Backups an, ab Stufe M10 stehen Tools zur Verfügung, die Metriken in Echtzeit anzeigen, automatisch archivieren, Empfehlungen zur Leistungsoptimierung erstellen und langsame Datenbankabfragen zur Diagnose anzeigen. In der Entwicklungsphase haben wir uns für die kostenlose Stufe entschieden, da die Funktionen und Leistung dafür ausreichen. Sollte das Produkt auf den Markt gehen, werden wir die kostengünstigste Stufe wählen, um die Option von Backups zu erhalten. Wenn das Projekt erfolgreich ist und wir viele Nutzer anziehen, wird flexibel, abhängig von benötigter Leistung, eine höhere Stufe gewählt.
Sowohl die Produktions, als auch die Entwicklungsumgebung werden als eigene Datenbanken von MongoDB Atlas gehosted. Dies verringert das Risiko von Code, der auf der lokalen Maschine funktioniert, aber auf der Produktionsumgebung Fehler wirft (\glqq It works on my machine\grqq). durch gleiche Werkzeuge und gleicher Technologie wird die Werkzeuglücke verringert und dementsprechend die Dev-Prod-Vergleichbarkeit erhöht. \cite{12FA1}

\subsubsection{Fazit}
%Warum für Mongo entschieden
MongoDB eignet sich sehr gut für unser Projekt, da...

\section{Schemata}
\subsection{Nutzer}
\subsection{Passwort}
\subsection{Sprache}
\subsection{Like}
\subsection{Chat}
\subsubsection{Nachricht}

\subsubsection{Database-as-a-Service}
Uns Fehlen die Kapazitäten und die Infrastruktur, um selbst die Datenbank zu betreiben. Stattdessen sind wir auf eine DataBase-as-a-Service-Lösung angewiesen. Diese lassen sich schnell einrichten, fallen selten aus und werden automatisch mit Updates versorgt. Es stehen Reportingtools zur Verfügung, die das Auswerten der Datenbank erleichern. DBaaS spart viel administrativen Aufwand und verringert damit die Kosten.
MongoDB Inc. bietet mit MongoDB Atlas eine Database-as-a-Service Lösung an, die Flexibel auf die Größe und Auslastung des Projektes angepasst werden kann. Dazu gibt es verschiedene Datenbankstufen, die mit höheren Kosten mehr Rechenleistung und weitere Funktionen erhält. Zwischen den Stufen kann flexibel gewechselt werden, um den aktuellen Anforderungen gerecht zu werden. Kostenpflichtige Stufen bieten die Möglichkeit von Backups an, ab Stufe M10 stehen Tools zur Verfügung, die Metriken in Echtzeit anzeigen, automatisch archivieren, Empfehlungen zur Leistungsoptimierung erstellen und langsame Datenbankabfragen zur Diagnose anzeigen. In der Entwicklungsphase haben wir uns für die kostenlose Stufe entschieden, da die Funktionen und Leistung dafür ausreichen. Sollte das Produkt auf den Markt gehen, werden wir die kostengünstigste Stufe wählen, um die Option von Backups zu erhalten. Wenn das Projekt erfolgreich ist und wir viele Nutzer anziehen, wird flexibel, abhängig von benötigter Leistung, eine höhere Stufe gewählt.
Sowohl die Produktions, als auch die Entwicklungsumgebung werden als eigene Datenbanken von MongoDB Atlas gehosted. Dies verringert das Risiko von Code, der auf der lokalen Maschine funktioniert, aber auf der Produktionsumgebung Fehler wirft (\glqq It works on my machine\grqq). durch gleiche Werkzeuge und gleicher Technologie wird die Werkzeuglücke verringert und dementsprechend die Dev-Prod-Vergleichbarkeit erhöht. \cite{12FA1}

\section{Avatarbilder}
Statt Bilddateien für Avatare direkt auf der Datenbank zu speichern, speichern wir nur URIs zu den Bildern auf unserer Datenbank. Die Bilder selbst werden auf AWS S3 gehosted, einem Speichersystem, welches für BLOB-Dateien optimiert ist. Dies nimmt der Datenbank Last ab und erhöht die Geschwindigkeit. Entsprechend mussten wir den S3-Speicher so einrichten, dass unser Backend die Berechtigung hat, Dateien zu erstellen und zu löschen. Unsere Datenbank wird mithilfe von GraphQL angesprochen, für S3 hat sich diese Lösung jedoch nicht angeboten. Für das Hochladen von Profilbildern mussten wir entsprechend eine weitere Route im Backend erstellen, die es Nutzern ermöglicht, Bilder hochzuladen. Wir entschieden uns für die npm-Pakete Multer und Multer-S3, welche uns erlauben, zu kontrollieren, ob es sich bei der gewählten Datei um eine Bilddatei handelt und ob diese eine bestimmte Bildgröße nicht übersteigt. 




\newpage
\section{Backend}\label{kap_Backend}
wofür backend

\subsection{Node}
NodeJS ist eine JavaScript Laufzeit, welche es ermöglicht, JavaScript Code auf einem Server auszuführen. Dieser Server beantwortet die Anfragen, welche der Nutzer über das Frontend verschickt und verwaltet die Routen der Domäne, um den gewünschten Inhalt anzuzeigen. Mit der Hilfe von Paketverwaltungswerkzeugen wie npm ist es möglich, Programmierpakete in das Projekt einzubinden.

Wie in \textit{package.json} definiert, handelt es sich bei der Datei \textit{server.js} um das Main-Programm. Wenn der Kommandozeilenbefehl \textit{node start} ausgeführt wird, wird der in \textit{server.js} definierte Quellcode ausgeführt.

Nachdem benötigte Pakete geladen wurden, werden mit der Hilfe von \textit{dotenv} Umgebungsvariablen definiert. Dies hat den Vorteil, dass privat zu haltende Informationen wie beispielsweise Passwörter und Schlüssel in einer weiteren Datei ausgelagert werden können, welche nicht durch git verfolgt wird. Dies erhöht die Sicherheit, da sich sonst unautorisierte Personen Zugriff zum System machen könnten. Außerdem können damit für Entwicklungs und Produktionsumgebung verschiedene Umgebungsvariablen verwendet werden, ohne dass verschiedene Zweige im git-System notwendig sind.

Danach verbindet sich der Node-Server mit der Datenbank, welche über MongoDB Atlas gehosted wird. Das verwendete Mongoose-Paket erlaubt mit Hilfe einer URI, welche den Anmeldeschlüssel enthält, die Autorisierung des Servers. Nachdem der Server erfolgreich verbunden wurde, wird geprüft, ob es sich bei der Node-Umgebung um die Entwicklungsumgebung handelt. In diesem Fall werden potenziell Pseudo-Daten für die Datenbank erstellt, um es dem Entwickler zu vereinfachen zu prüfen, ob sein Quellcode funktioniert.

Zudem werden die Express- und Apollo-Applikation gestartet und der Server danach endgültig aktiviert.

\subsection{Express}
HTTP-Anfragen an den Server werden durch Express verwaltet. Dabei gibt es Weiterleitungen (Routen) und Middleware.

\subsubsection{Middleware}
Middlewarefunktionen sind Funktionen, die als Teil einer HTTP-Anfrage ausgelöst werden, wenn sie für die angefrage Route aktiviert sind. Middlewarefunktionen werden hintereinander verkettet, sobald die erste Funktion ausgeführt wurde und keinen Fehler zurückgegeben hat, wird die nächste Middlewarefunktion ausgeführt, bis zum Schluss die Routenfunktion antwortet. Middlewarefunktionen können genutzt werden, um Protokoll über die Webseitenzugriffe zu führen, Anfragen zu autorisieren oder Fehlerhafte Anfragen zu blockieren.

\paragraph{CORS}
Um Anfragen des Frontends an das Backend zu erlauben, ist CORS (Cross-Origin Resource Sharing) notwendig. Per Same-Origin-Poicy (SOP) wären Zugriffe von einer Domäne auf die andere normalerweise nicht gestattet, CORS ermöglicht es, Anfragen vom Frontend - welches eine andere Domäne besitzt - zu ermöglichen. Die Miskonfiguration von CORS gefährdet die Integrität der Webseite, es ist daher ratsam, nur für spezielle Domänen - in diesem Fall die Domäne des Frontends - CORS zu erlauben. Aktuell ist CORS für alle Anfragen erlaubt, um die Webseite sicherer zu gestalten muss in Zukunft die CORS-Richtlinie restriktiver eingestellt werden. Wenn die CORS-Regeln verletzt werden, wird eine Fehlermeldung ausgegeben und keine weiteren Funktionen der Kette zum Beantworten der HTTP-Anfrage ausgeführt.

\paragraph{JWT}
Einige Daten der Schnittstelle sind öffentlich. Andere Daten, wie zum Beispiel private Nutzerdaten, benötigen stattdessen eine Anfrage des jeweiligen Nutzers. Um die Autorität des Nutzers zu bestätigen wird beim Anmeldeprozess ein JSON Web Token (JWT) nach RFC 7519 Standard an den Klienten verschickt\cite{RFC7519}. Bei allen Folgeabfragen, so auch einer Abfrage der privaten Nutzerdaten, wird das Token als Teil des Anfragekopfes (request header) an das Backend gesendet. Durch Middleware wird das Token kontrolliert und der Nutzer authentifiziert und autorisiert. Das entschlüsselte Token wird an die folgenden Funktionen weitergeleitet und kann von diesen genutzt werden, um den anfragenden Benutzer herauszufinden. Sollte zum Beispiel eine Anfrage an die GraphQL-Schnittstelle gesendet werden, wird zuerst das JWT Token entschlüsselt und der Nutzer in die Abfrage hinzugefügt. Ohne die Middleware wäre die Datenbankschnittstelle, welche das Profil des aktuellen Nutzers anzeigt, nicht verwendbar.

\subsubsection{Routen}
Routen geben an, welcher Inhalt bei einer HTTP-Anfrage zurückgegeben werden soll.

\paragraph{Root ("/")}
Um zu testen, ob das Backend generell auf Abfragen reagiert, soll zunächst die Antwort bei der Abfrage des Stammverzeichnisses beantwortet werden.
\begin{lstlisting}
    app.get("/", (request, response) => {
        response.send("Hello")
    })
\end{lstlisting}
Bei einer GET-Anfrage des Stammverzeichnisses wird in diesem Fall mit \textit{"Hello"} geantwortet.

\paragraph{/graphql}
Eine Anfrage an die Route \textit{/graphql} wird von Apollo, dem gewählten Framework für GraphQL beantwortet. Der Großteil aller Anfragen an das Backend gehen an diesen Endpunkt. Genauere Informationen zu GraphQL und Apollo sind in dem entsprechenden Kapitel zu finden. Durch die gewählten Optionen ist der GraphQL-Playground aktiviert, eine graphische Oberfläche die es einem benutzer der API ermöglicht, seine Anfragen direkt auf der Webseite in einem Texteditor zu schreiben. Durch das Erlauben von Introspektion steht zudem eine Umfangreiche Dokumentation der möglichen Anfragen und dessen Datentypen zur Verfügung, welche es dem Entwickler vereinfachen, seine eigenen Anfragen zu testen.

\paragraph{/api}
Das Hochladen von Bildern ließ sich über GraphQL nicht zur Zufriedenheit lösen. Es wurde daher eine klassische, REST-basierte Lösung entwickelt, welche es mit Hilfe von Multer - einer Node Middleware zum behandeln von Multipart-Daten, worunter Bilddateien zählen - ermöglicht, Bilddateien auf AWS hochzuladen und die URI des hochgeladenen Bildes in MongoDB zu speichern. Dazu werden mit Hilfe des \textit{aws-sdk} Paketes die verwendeten Umgebungsvariablen für die Autorisierung bei AWS und der Wahl der Region und Schnittstellenversion verwendet. 

Wenn ein Nutzer ein Profilbild hochladen will, wird dafür Subroute \textit{/avatar}, im vollen also \textit{/api/avatar} verwendet. Durch verschiedene Regeln wird mit Hilfe von Multer getestet, ob es sich bei der vom Nutzer gesendeten Ladung des Paketes (Payload) um eine Datei handelt, diese in einem akzeptierten Bildvormat ist und die maximale Dateigröße nicht überschritten wurde. Sollten alle Tests erfolgreich sein, wird das Bild mit einem durch UUID Version 4 generierten Namen auf S3 hochgeladen. Um das Bild öffentlich anbieten zu können, wird die Zugriffskontrollliste auf öffentlich lesbar gestellt. Schlussendlich wird der durch S3 generierte Link der Datei auf der Datenbank dem entsprechenden Nutzer zugewiesen.

Die Route \textit{/avatar} steht nur für POST, aber nicht für GET Anfragen zur Verfügung. Für das Abfragen des Profilbilds wird stattdessen die GraphQL-Schnittstelle verwendet, die auch sonst für die meisten Abfragen verwendet wird. GET Anfragen der Bilddateien stellen kein Problem dar, da die Datenbank lediglich einen Link des Bildes speichert und S3 beim öffnen dieses Links das Bild zurück gibt.

Falls benötigt können in Zukunft weitere Subrouten für \textit{/api} erstellt werden, die mit oder ohne Multer oder auch mit Hilfe von völlig anderen Paketen - oder ohne Pakete - in der Lage sind, Auf Schnittstellenanfragen zu antworten. So könnte in einer späteren Version der Nutzer in der Lage sein, Bilder in Nachrichtenverläufen zu versenden, in diesem Fall würde, ähnlich wie bei \textit{/avatar} die Route \textit{/chat} mit Hilfe von AWS S3 und Multer zum Hochladen von Bildern dienen.



NODE
Models - Die in der Datenbank definierten Schemata sind in Mongoose definiert.

\newpage
\section{GraphQL-Schnittstelle}\label{kap_Schnittstelle}
\paragraph{}
GraphQL ist eine Abfragesprache und Server-Laufzeitumgebung für APIs. Diese dient dazu, die Daten zu liefern, die anfordert werden.
\\\\
Im Hinblick auf die Art und Weise, wie Abfragen an den Server mithilfe von GraphQL behandelt werden können, sind folgende Aspekte zu beachten.
\subsubsection*{Eigenschaften}
  \begin{itemize}
    \item
          GraphQL-Aufrufe werden in einem einzigen Round Trip gehandhabt.% Wir bekommen genau die Daten, die angefragt haben (kein Over-Fetching).

    \item
          Stark definierte Datentypen, welche das Risiko einer Fehlkommunikation zwischen Client und Server verringern.

    \item
          Eine Anwendungs-API kann sich mit GraphQL weiterentwickeln, ohne dass bestehende Abfragen beeinträchtigt werden.
    \item
          GraphQL schreibt keine spezifische Anwendungsarchitektur vor. Es kann auf einer vorhandenen REST-API installiert und mit aktuellen API-Management-Tools verwendet werden.
    \item
          Als Alternative zu REST ermöglicht GraphQL Entwicklern die Erstellung von Abfragen zur Extraktion von Daten aus mehreren Quellen mit einer einzigen API-Abfrage.

  \end{itemize}
  
  \subsection{Anforderungen und Technologiewahl}
Die Webseite besteht aus statischen und dynamischen Inhalten.
Statische Inhalte wie beispielsweise die Menuführung, liegen offen in einem HTML-Dokument und werden bei Abfrage der Webseite gesendet. 
Dynamische Inhalte, unter anderem Profile, Chats und Nachrichten, werden mit Daten aus der Datenbank gefüllt.
Um diese Daten bereitstellen zu können, ist eine Schnittstelle zwischen Datenbank und Frontend notwendig. 

Es wurde sich für GraphQL statt einer standardmäßigen REST-Schnittstelle entschieden.
GraphQL bietet verschiedene Verbesserungen, wie zum Beispiel Leistungsoptimierungen durch Overfetching und Underfetching, Stabilität durch ein Typenschema und Selbstdokumentation.

\subsection{Overfetching und Underfetching}
\paragraph{}
Antworten von REST-Schnittstellen sind dafür bekannt, zu viele Daten (Overfetching) oder zu wenige Daten (Underfetching) zu beinhalten.
In beiden Fällen entstehen - durch mehr benötigte Bandbreite oder mehr Anfragen - Leistungseinbuße, die als Folge längere Antwortzeiten und höhere Serverkosten beinhalten. \\  %TODO cite
Das Problem ist mit REST-Schnittstellen schwer zu umgehen.
Eine perfekte Konfiguration würde exakt die Endpunkte beinhalten, die genau die angefragen Daten liefern.
Dies ist jedoch praxisfern, da sich die Anforderungen in einem Projekt ständig verändern und jeder Endpunkt für dessen Betrieb Ressourcen benötigt und gewartet und getestet werden muss.

\paragraph{}
GraphQL löst das Problem, indem bei jeder Anfrage genau definiert wird, welche Daten von welchen Endpunkten erfragt werden.
Dies entlastet die Datenbank, da diese weniger Daten zur Verfügung stellen muss.
Auch wird die Arbeit des Frontends erleichtert, da nur eine einzige Abfrage erstellt werden muss und irrelevante Daten gar nicht erst abgefragt und somit nicht geliefert werden müssen.\\
REST fokussiert sich auf die verschiedenen Endpunkte, während GraphQL den Fokus auf die einzelnen Aufgaben legt.
In GraphQL entscheidet die Applikation über die Daten die es erhält, nicht der Server.
Mehrere Abfragen können in eine einzige Schnittstellenabfrage gebündelt werden, der Client erhält somit zusammenhängende Daten und die Datenbank wird durch präzise Abfragen geschont.

\subsection{Stabilität}
Durch GraphQls Typenschemata wird definiert, welche Abfragen und Mutationen möglich sind.
Sollte bei einer Abfrage die Datenbank einen anderen Datentyp als durch das Typenschema definiert liefern, wirft GraphQL einen Fehler.
Der Klient kann somit bei einer Abfrage damit rechnen, Daten in den angegebenen Datentypen zu erhalten, Ausnahmen für andere Datentypen müssen nicht definiert werden.
Genauso wird eine Abfrage mit nicht erlaubten Datentypen direkt mit einer Fehlermeldung beantwortet, die genau auf den falschen Datentyp hinweist.
Dadurch werden Inkonsistenzen der Datenbank oder weitere Datentypenkontrollen durch Design bereits stark mitigiert.
%TODO cite

\subsection{Dokumentation}
GraphQL selbstdokumentiert jeden Endpunkt durch die vorliegenden Datentypen.
Bei Änderungen an der Schnittstelle wird die Dokumentation automatisch aktualisiert. Auch kann manuell eine Beschreibung des Endpunkts hinzugefügt werden. Dies erleichtert es dem Frontend, die richtigen Endpunkte und Datentypen für Abfragen und Mutationen zu verwenden. Durch die angegebenen Datentypen ist das Frontend in der Lage, direkt zu erkennen, welche Datentypen in der entsprechenden Abfrage oder Mutation zu verwenden sind und kann damit Zeit sparen.
Gleichzeitig fällt Zeit weg, die für das manuelle Dokumentieren der Schnittstelle angefallen wäre.

\subsection{GraphiQL}
Mit GraphiQL liefert GraphQL einen Webeditor, der das Schreiben von Abfragen und Mutationen vereinfacht.
Der Editor liefert einige Funktionen, die den Umgang einfacher machen, wie zum Beispiel automatisches Komplettieren von Variablennamen oder das Formatieren in eine lesbarere Struktur.
Der Editor hat außerdem Zugriff auf die Typenschemata und Dokumentation und kann damit Abfragen auf syntaktische Fehler untersuchen, noch bevor diese abgeschickt werden.
Der Editor steht unter der Route \textit{/graphql} für jeden Nutzer zur freien Verfügung und erleichtert somit die Arbeit für alle Programmierer, die unsere Schnittstelle verwenden.
Im Projekt konnte durch die Nutzung des Editors beim Erstellen von Datentypdefinitionen, Abfragen und Mutationen Programmierzeit gespart werden.

\begin{figure}
	\centering
    \includegraphics[width=\textwidth]{sources/graphiql.png}\cite{}
	\caption{GraphiQL Benutzeroberfläche. Links: Zwei Endpunkte werden durch eine Abfrage gebündelt abgefragt. Mitte: Antwort der Endpunkte. Rechts: Dokumentation und Typendefinition des Endpunkts \textit{userSelf}.}
	\label{fig:gql:1}
\end{figure}

\subsection{Resolver}
Durch die Pakete \textit{graphql-compose} und \textit{graphql-compose-mongoose} werden aus den bestehenden Datenbankmodellen Typenschemata entsprechend der Datentypen erstellt, die in GraphQL weiterverwendet werden können. 
Resolver ("to resolve", etw. auflösen/abwickeln/klären) beantworten Abfragen und Mutationen.
Sie bilden das Bindeglied zwischen Schnittstellenabfrage und Datenbankantwort.
So wird beispielsweise eine Schnittstellenabfrage, welche nach dem Namen, Alter und Freitext des Nutzers "Foobar" fragt, durch einen Resolver in die entsprechende Datenbankabfrage übersetzt und die Antwort der Datenbank wiederum so editiert, dass diese der gewünschten Datenstruktur der Abfrage entspricht. %TODO schaubild stattdessen
Simplere Resolver, welche zum Beispiel einen einzelnen Datenbankeintrag durch die ID identifizieren (\textit{findOne}) oder anhand der bestehenden Datenstruktur einen Datenbankeintrag erstellen (\textit{createOne}), werden durch die verwendeten Pakete bereit gestellt und lassen sich in das Projekt integrieren.
Für komplexere Resolver lassen sich bestehende Resolver anpassen oder gänzlich neue erstellen.
Die verwendeten Resolver und Typenschemata werden dem \textit{schemaComposer} bereit gestellt, welcher aus den bestehenden Daten die Dokumentation generiert.


\subsection{Verwendete Abfragen und Mutationen}
\paragraph{}
Nachfolgend werden einige der verwendeten Abfragen und Mutationen exemplarisch vorgestellt.

\paragraph{}
Für den Registrierungsvorgang wird die Mutation \textit{signup} verwendet.
Durch die Angabe von Email, gewünschtem Namen und Passwort wird ein Nutzerkonto erstellt.
In Zukunft kann durch Angabe von Namen und Passwort mit der \textit{login} Abfrage ein Authentifizierungstoken erstellt werden. Das Token wird bei Abfragen, die eine Authentifizierung oder Authorisierung verlangen, verwendet.

\begin{figure}
	\centering
    \includegraphics[width=\textwidth]{sources/graphiql_signup.png}
	\caption{Registrierung des neuen Nutzers \textit{MyName}. Eigene Darstellung.}
	\label{fig:gql:2}
\end{figure}

\begin{figure}
	\centering
    \includegraphics[width=\textwidth]{sources/graphiql_login.png}
	\caption{Anmeldung des Nutzers. Die Antwort enthält das JSON Web Token, welches für bestimmte Abfragen zur Authentifizierung und Autorisierung dient. Eigene Darstellung.}
	\label{fig:gql:3}
\end{figure}

\paragraph{}
Alle weiteren vorgestellten Endpunkte benötigen zur eindeutigen Indentifizierung ein valides Authentifizierungstoken.

\paragraph{}
Um das eigene Profil anzuzeigen, wird die \textit{userSelf} Abfrage verwendet.
Diese zeigt Daten entsprechend des Datenbankschemas an.
Einige dieser Daten sind veränderbar, so zum Beispiel der Freitext.
Zum Verändern der Daten ist die Mutation \textit{userUpdateSelf} vonnöten.

\begin{figure}
	\centering
    \includegraphics[width=\textwidth]{sources/graphiql_userUpdateSelf.png}
	\caption{Das Geschlecht, Sprachen, Rolle etc. für den aktiven Nuter werden gesetzt. Unten: Ausschnitt des Autorisierungstokens. Durch das Token wird der aktive Nutzer identifiziert. Eigene Darstellung.}
	\label{fig:gql:4}
\end{figure}

Zum Finden von neuen Nutzern wird die Abfrage \textit{userManyToSwipe} verwendet. Dabei werden dem abfragendem Nutzer durch einen Algorithmus die Profile von mehreren Nutzern geladen, die sich nicht bereits in der Freundesliste befinden oder vom aktiven Nutzer blockiert wurden.
Nutzer, welche dem aktiven Nutzer eine Freundesanfrage geschickt haben, werden präferiert gezeigt, um die Chance neuer Freundschaften zu erhöhen.
Es ist möglich, durch Filter die Menge der potenziellen Nuter zu reduzieren.
Im Frontend werden die Profile nacheinander angezeigt.
Es wäre möglich gewesen,  mit jeder Abfrage nur einen einzelnen Nutzer anzuzeigen.
Da das Laden von mehreren Nutzern pro Abfrage jedoch die Zahl der Abfragen stark erhöht, wurde sich dagegen entschieden.
Mit der Mutation \textit{swipe} kann dann eine Freundschaftsanfrage an den angegebenen Nutzer geschickt werden.

\begin{figure}
	\centering
    \includegraphics[width=\textwidth]{sources/graphiql_userManyToSwipe.png}
	\caption{Abfrage nach mehreren Nutzern mit userManyToSwipe. Durch den Filter werden nur Nutzer im Alter zwischen 20 und 30 Jahren angezeigt. Eigene Darstellung.}
	\label{fig:gql:5}
\end{figure}

\begin{figure}
	\centering
    \includegraphics[width=\textwidth]{sources/graphiql_swipe.png}
	\caption{Der aktive Nutzer versendet einen Like an den durch \textit{recipient} definierten Nutzer. Eigene Darstellung.}
	\label{fig:gql:6}
\end{figure}

Die Nachrichten eines Chats können durch die Abfrage \textit{getChat} gelesen werden.
Dazu ist die ID des Chat anzugeben.
Der optionale Parameter „Seite“ gibt an, welche Nachrichten zurückgegeben werden sollen.
Auf jeder Seite befinden sich 20 Nachrichten, beginnend bei \enquote{Seite=1} mit den neuesten 20 Nachrichten.
Sollte der Seitenparameter nicht angegeben werden, wird nur die neueste Nachricht geladen.
Dies wird bei der Vorschau des Chats für die Freundesliste verwendet.

\begin{figure}
	\centering
    \includegraphics[width=\textwidth]{sources/graphiql_getChat.png}
	\caption{Durch den Parameter \textit{page: 1} werden die bis zu 20 letzten Nachrichten geladen. Da der Chat nur 4 Nachrichten beinhaltet werden nur diese geladen. Unten: Das Authentifizierungstoken \textit{x-auth-token} befindet sich in dem HTTP-Header der Abfrage. Eigene Darstellung.}
	\label{fig:gql:7}
\end{figure}

Mit \textit{sendMessage} können Nachrichten versendet werden. Dazu ist die ID des Chats notwendig.
Der im Parameter \textit{Inhalt} definierte Text wird dann in den Chat als neueste Nachricht versendet.
Um eine Nachricht zu editieren  oder zu löschen, ist die Mutation \textit{editOrDeleteMessage} zu verwenden.
Zusätzlich zur Chat-ID ist die Nachrichten-ID der zu verändernden Nachricht anzugeben. Der optionale Parameter \textit{content} gibt im Falle einer Änderung den neuen Nachrichteninhalt an.
Sollte \textit{content} nicht definiert sein, wird die gewählte Nachricht stattdessen gelöscht. 

\begin{figure}
	\centering
    \includegraphics[width=\textwidth]{sources/graphiql_editMessage.png}\cite{}
	\caption{Die bestehende Nachricht (Siehe Abfrage \textit{getChat}) wird durch den in \textit{content} definierten Text ersetzt. Eigene Darstellung.}
	\label{fig:gql:8}
\end{figure}

\begin{figure}
	\centering
    \includegraphics[width=\textwidth]{sources/graphiql_deleteMessage.png}\cite{}
	\caption{Die soeben editierte Nachricht wird gelöscht. Eigene Darstellung.}
	\label{fig:gql:9}
\end{figure}

\subsection{Fazit}
Es konnte gezeigt werden, dass GraphQL eine geeignete Wahl für das Projekt ist.
Neben den Anforderungen, welche auch durch eine REST-Schnittstelle gelöst werden könnten, glänzt GraphQL mit den genannten Vorteilen wie Leistungsoptimierungen, Stabilität und Selbstdokumentation.
Die gewählten Resolver ermöglichen einen einfachen Zugriff auf die auf der Datenbank liegenden Informationen und GraphiQL ermöglicht technikaffinen Nutzern einen tieferen Umgang mit der Webseite.
Das Hochladen von Bilddateien stellt GraphQL vor eine Herausforderung, dieses Problem lässt sich jedoch mit dem zurückgreifen auf klassische REST-API lösen.

\newpage
\section{Frontend}\label{kap_Frontend}
% Wie werden die Informationen die Benutzer gezeigt und wie können sie diese manipulieren
%\paragraph{}
\begin{flushleft}
In diesem Kapitel werden Angular, Vue und React für die Entwicklung des Frontends bewertet. 
\\
Alle drei Technologien haben unter anderem die Vorteile, gut dokumentiert zu sein und große Gemeinschaften zu haben. Dies begünstigt den Lernprozess und verschleunigt mögliche Problembehebungen.

\begin{flushleft}
Sie haben die Unterstützung von Unternehmen wie Facebook, Google und Alibaba. Sie sind komponentenbasiert, was ihre Testbarkeit erleichtert.
\\
Es wurden die oben genannten Technologien ausgewählt, weil diese an den ersten drei Stellen in diversen Ranglisten auftauchen{\cite{SO01}}.
\end{flushleft}

  Folgende Kriterien wurden berücksichtigt:
  \begin{itemize}
    \item
          Der Schwierigkeitsgrad, um die Technologie zu lernen. 
          Die Lernkurve ist wichtig, weil das Entwicklungsteam begrenzte Zeit für die Entwicklung des Projekts hat. 
    %\item
    % Die Flexibilität des Frameworks.
    % Der Zwang, Probleme auf eine bestimmte Weise zu lösen, oder die Alternative, eigene Lösungswege zu finden.

  %  \item
  %  Anzahl der ungelösten Probleme\footnote{Open issues}. 

    \item
          Die Akzeptanz bei den Entwicklern und die Anzahl von heruntergeladenen NPM-Paketen sind relevant, weil vermehrte Nutzung dazu führen kann, dass Probleme schneller identifiziert und gelöst werden.{\cite{LIN1}}

          %da Probleme generell schneller identifuíyiert und potentiell gelost werden, wenn mehr menschen mit fachwissen mit den technologien Arbeiten. 
          
          %weil je mehr Nutzer ein Framework benutzen, desto wahrscheinlicher ist es, dass eine Lösung für mögliche Probleme gefunden wird. 
      
    \item
          Persönliche Präferenz und Vorkenntnisse des Entwicklungsteams. 
  \end{itemize}

\end{flushleft}

\newpage    
\subsection{Popularität und Statistiken}
Die Informationen unten geben Hinweise auf den Nutzen und die Beliebtheit der Open-Source-Projekten React, Angular und Vue. Diese Indikatoren sind wichtig, denn je mehr Menschen an einem Open-Source-Projekt beteiligt sind, desto mehr Menschen tragen dazu bei, Fragen zu beantworten und Probleme\footnote{Code issues} zu lösen {\cite{LIN1}}.
\begin{flushleft}
  \textbf{Developer Survey 2021 - StackOverFlow}\\
  StackOverFlow ist die größte Gemeinschaft von Softwareentwicklern, wo Wissen und Fragen zur Softwareentwicklung ausgetauscht wird.

  Seit 2011 führt StackOverFlow eine jährliche Umfrage zur Softwareentwicklung durch.

\begin{figure}[h]
  \centering
  \includegraphics[scale=0.5]
  {Most popular Web Frameworks [Both] 2021.png}
  \caption{ Which web frameworks and libraries have you done extensive development work in over the past year, and which do you want to work in over the next year? (If you both worked with the framework and want to continue to do so, please check both boxes in that row.) {\cite{SO01}}}

\end{figure}
\end{flushleft}

\newpage

\begin{flushleft}
  \textbf{Github}\\
  GitHub ist die Plattform für die Versionskontrolle von Source-Code, wo mehr als 238 Millionen Repositories\footnote{Ablage wo Quellkode gespeichert wird} verwaltet werden{\cite{GH07}}.  
\end{flushleft}

Die Anzahl der Repositories, in denen Code für Angular, React und Vue gespeichert wird, gibt einen Hinweis auf die Beliebtheit dieses Frameworks.
\\
\begin{table}[h!]
  \centering
  \begin{tabular}{ |p{3cm}||p{3cm}|p{3.6cm}|p{3.6cm}|  }
    \hline
    \multicolumn{4}{|c|}{Github Statistiken}                        \\
    \hline
                            & Angular/core & Vue       & React     \\
    \hline
    Anzahl von     Repositories & 2,011,663     & 2,299,614 & 7,868,546
    \\
 %   \hline
  %  Offene Issues           & 1,716         & 321       & 634
  %  \\
 %   \hline
  %  Stars   \footnote{Stars werden auf Github verwendet, um die Popularität eines Repositorys anzuzeigen.}                 & 77.2k         & 190k      & 170k
   % \\
    \hline
  \end{tabular}
\end{table}

{\cite{GH04, GH05, GH06}}

%https://risingstars.js.org/2020/es#section-statemanagement
%https://www.freecodecamp.org/news/angular-react-vue/

\textbf{NPM Paketen}\\
NPM verwaltet die Abhängigkeiten eines Angular, React oder Vue Projekts.

Die Anzahl der heruntergeladenen NPM-Pakete gibt einen Hinweis auf die Nutzung der Technologie.

\begin{center}
  \includegraphics[scale=0.4]{sources/NPM-Trends React_Angular_Vue}\label{fig:NPM-Trends React_Angular_Vue}\\
  \textbf{Abbildung \autoref{fig:NPM-Trends React_Angular_Vue}:} heruntergeladenen NPM-Pakete \\@angular/core vs react vs vue
    {\cite{NPM01}}
\end{center}

\textbf{Google Trends}\\
Laut Google Trends war React zwischen 01.11.2020 und 26.10.2021 das am häufigsten konsultierte Frontend Technologie in Deutschland.
\\
Diese Metrik ist ein weiterer Aspekt für die Beliebtheit von Frameworks
\begin{center}
  \includegraphics[scale=0.5]{sources/GoogleTrends Vue React Angular 1.11.2020 26.10.2021}\label{fig:GoogleTrends Vue React Angular 1.11.2020 26.10.2021}\\
  \textbf{Abbildung \autoref{fig:GoogleTrends Vue React Angular 1.11.2020 26.10.2021}:} Google Trends Angular vs React vs Vue
    {\cite{GO01}}
\end{center}

\textbf{Jobangebote}\\
Als angehende Informatiker sind berüfliche Perspektiven relevant, deswegen wurde den Arbeitsmarkt in der Entscheidungsfindung berücksichtigt.

%In der Entscheidungsfindung sind ebenfalls künftige Perspektiven relevant.
%Aus diesem Grund ist als angehende Informatiker den Arbeitsmarkt berücksichtigt.    
\begin{flushleft}
Anzahl der Jobangebote bei LinkedIn in Deutschland:
\\
Angular: 7.657 Ergebnisse{\cite{LI3}}
\\
React: 11.523 Ergebnisse{\cite{LI2}}
\\
Vue: 2.800 Ergebnisse{\cite{LI1}}
\end{flushleft}

\subsection{Angular}
Angular ist das TypeScript Entwicklungsplattform von Google für Mobil- und Webanwendungen.


Angular hat eine steile Lernkurve im Vergleich zu React und Vue.{\cite{E01}}, S.25.
\\
Angular lässt weniger Spielraum für eigene Entscheidungen darüber, wie der Code entwickelt wird. Deshalb eignet sich Angular besser für Projekte, wo mehrere Entwickler zusammenarbeiten. 

%Der Code wird mit Angular nach der Ri in einer vorgegebenen Art geschrieben werden sollte, um Einheitlichkeit zu schaffen.

Zu beachten ist, dass Angular nicht JS als Hauptprogrammiersprache verwendet, sondern TypeScript. Dies würde einen zusätzlichen Aufwand für das Erlernen des Frameworks bedeuten.

%Angular ist ein komplexeres Framework als React und Vue.

\subsection{Vue}
Von den drei Frameworks ist Vue das einzige, das nicht von einem Unternehmen entwickelt wurde. Der Entwickler Evan You hat bei Google gearbeitet und sich mit AngularJS beschäftigt. Das Framework wird von einer Entwicklergemeinschaft gepflegt. Es wird jedoch von Unternehmen wie Alibaba, Nindendo, Expedia und anderen genutzt.
...
%\\

%CHECK THIS
%https://merehead.com/blog/angular-vs-react-vs-vue-best-choice-2022/


{\cite{V02}}
%Google or Facebook on the other side can almost ensure a long-term commitment both in terms of active development and nancial aspects to their respective frameworks. S.59

%PASSENTHES KAPITELTHEMA´?
AB HIER ÜBERARBEITEN:
\\

\subsection{React}
Bei der endgültigen Entscheidung wurden Vue und React aufgrund ihrer geringen Lernkurve, ihrer guten State Management, ihre große Popularität und ihrer Einfachheit berücksichtigt.

Die Entscheidung fiel für React aufgrund der Vorkenntnisse des Entwicklerteams, der Anzahl an Jobangeboten und der Unterstützung durch ein Unternehmen wie Facebook.

%Die Entwicklung mit Angular hätte mehr Zeit und Lernaufwand gekostet. Außerdem ist das Projekt nicht komplex genug, um das Angular-Ökosystem zu benötigen.

\subsubsection{Vorteile}
\textbf{Deklarativ} \\
Mit React ist es möglich, interaktive Benutzeroberflächen, Ansichten für jeden Zustand der Anwendung zu erstellen. %React aktualisiert und rendert effizient genau die richtigen Komponenten, wenn sich deren Daten ändern.
Durch deklarative Ansichten wird der Code vorhersehbarer und einfacher zu debuggen.
\newline

\textbf{Komponentenbasiert}\\
React ist auf gekapselte Komponenten basiert, die ihren eigenen Zustand verwalten.
%Da die Komponentenlogik in JavaScript geschrieben wird, kann man umfangreiche Daten durch die Anwendung leiten und den Zustand aus dem DOM heraushalten.
\newline

\textbf{Große Entwickler-Community}\\
React besteht aus rund 56.162 professionellen Entwicklern auf der ganzen Welt.
Laut einer StackOverFlow Umfrage hat React.js im Jahr 2021 jQuery als das am häufigsten verwendete Web-Framework überholt. {\cite{SO01}}
\newpage

\subsubsection{Nachteile}
%Bei der Auswahl des Frameworks wollten wir so unvoreingenommen wie möglich sein, daher listen wir einige Aspekte auf, die bei Projekten mit React zu beachten sind.

\textbf{JSX}\\
Während dies für einige Entwickler ein Nachteil sein könnte, ist es wichtig zu beachten, dass JSX auch seine Vorteile hat und hilft, den Code vor Injektionen zu schützen.{\cite{R02}}
\newline

\textbf{Ein hohes Entwicklungstempo}\\
Entwickler, die das Entwicklungstempo als Nachteil sehen, würden argumentieren, dass sie die Arbeit mit React ständig neu erlernen müssen und es schwierig ist, damit Schritt zu halten.

Es ist wichtig festzustellen, dass neue Entwicklungen des Frameworks verbessern und dazu beitragen, dass er ein höheres Leistungsniveau erreicht.
\newline

\textbf{Eine zu leichte Dokumentation}\\
Aufgrund der rasanten Entwicklung ist die Dokumentation in Bezug auf die neuesten Aktualisierungen und Änderungen oft spärlich.{\cite{R01}}

\subsection{React Hooks}
%https://www.javatpoint.com/react-hooks
Beginnend mit 16.8.0, enthält React eine stabile Implementierung von React Hooks.

Ab dieser Version wird React empfohlen, keine Klassen mehr für die Erstellung von Komponenten zu verwenden.
Mit Klassen geschriebene Komponenten werden weiterhin unterstützt und müssen nicht neu geschrieben werden.
{\cite{R05}}

\textbf{useState}
Der Hook useState bietet die Möglichkeit an, den Zustand einer Anwendung zu verwalten. Sie besteht aus mindestens einen Wert (value) und einer Funktion (updateValue), die den Wert aktualisiert.
%Der Wert kann ein Zahl, ein String, ein Array oder sogar ein Objekt sein.
%Optional kann ein Anfangswert (initialValue) definiert werden.

In unserem Projekt wurde useState verwendet, um alle Benutzerdaten zu manipulieren und dem Benutzer anzuzeigen.

Diese Benutzerdaten sind die Antwort auf die an den Server gesendete Abfrage.
Ein Beispiel von einer Server-Antwort befindet sich im Anhang 1.
Eine genauere Funktionsweise der Serverabfragen folgt in Kapitel Serverabfragen.
\\
%In späteren Kapiteln wird näher erläutert, wie diese Informationen, die durch einen einzigen Abfrage an den Server erhalten werden, in den Chat- und Benutzerkomponenten genutzt werden.

%Wie man sieht, gibt es innerhalb des states nicht nur einen Wert, sondern mehrere.
%Es handelt sich eigentlich um ein Objekt. Objekte sind dasselbe wie Variablen in JavaScript, der einzige Unterschied ist, dass ein Objekt mehrere Werte in Form von Eigenschaften und Methoden enthält.
%Objekte können andere Objekte, Strings, Arrays, Zahlen und so weiter enthalten.
\newpage

\textbf{useEffect}
useEffect ermöglicht es, verschiedene Arten von Effekten zu erzeugen, nachdem eine Komponente gerendert wurde.

Unter anderem ist mit useEffect folgenden möglich:\\\\
\\-Aktualisieren des DOM,
\\-Abrufen und Konsumieren von Daten von einer Server-Schnittstelle,
\\-Einrichten eines Abonnements, usw.
%https://www.javatpoint.com/react-hooks
\\
\begin{flushleft}
%  Der Hook useEffect entspricht einer Kombination aus componentDidMount, componentDidUpdate und componentWillUnmount.

  Bei diesem Projekt wird die Anzeige der Daten abhängig der Änderungen des lokalen Zustands manipuliert.
\end{flushleft}

Im Bereich Profil werden die Änderungen erstmal zwischengespeichert, bevor sie in der Datenbank gespeichert werden. 
\\
Die Abfrage zur Aktualisierung der Daten wird in dem Moment an den Server gesendet, in dem der Benutzer auf die Taste „Save/Speichern“ drückt.
\\
Zu diesem Zeitpunkt wird die Benutzeroberfläche durch useEffect aktualisiert, da sich eine der Abhängigkeitsvariablen geändert hat.
%\\
%Der globale Zustand „state“ ist in der Komponente App enthalten.
%In diesem wird die Antwort von der Datenbankenabfrage gespeichert.
%Wir definieren den Zustand als Abhängigkeit von useEffect, so dass er bei jeder Änderung in der Benutzeroberfläche aktualisiert wird.
\\

\begin{flushleft}
  Es ist möglich, mehrere useEffects zu einstellen und mit ihrer jeweiligen Abhängigkeiten separat definieren.
  Wenn es erfordelich ist, dass der Code innerhalb dem useEffect nur einmal nach dem ersten Rendering(deutsches WORT) ausgeführt wird, muss ein leeres Array als Abhängigkeit definiert werden.
\end{flushleft}
  
%WELCHES PROBLEM HABEN WIR DAMIT GELÖST?  

\textbf{useContext}
useContext ermöglicht Daten innerhalb bestimmte Komponenten zu teilen.

In diesem Projekt war useContext eingesetzt, um die Benutzerdaten und den Authentifizierungstoken über die Komponenten  „Profile“, „Chat“, „Users“ (to swipe) zu teilen.
\\
In frühen Versionen des Projekts wurde für den gleichen Zweck props verwendet.  {\cite{R03}}
UseContext bietet eine weitere Lösung, um Daten zwischen Komponenten auszutauschen.{\cite{R04}}
\\

Im Anhang 5 und 6 befidet sich der Code mit dem den Authentifizierungstoken und die Benutzerdaten global  mithilfe von useContext verwaltet wird.

\subsection{Fazit}
Es ist anzumerken, dass die Entwicklung des Projekts mit jedem der drei Frameworks möglich war.
...

\newpage
\section{Qualitätssicherung }\label{kap_QS}
\textbf{Software Testing 14.11}
%WARUM SOLLTE GETESTET WERDEN´?
%WIESO END2END UND UNIT TEST?
%PROJEKTE SCHREITERN WEGEN MANGELHAFTE TESTING \%

%In den letzten x Jahren hat d
Das Software-Testing hat in den letzten Jahren an Bedeutung gewonnen. 
Einer der Gründe dafür ist, dass das Risiko, gar nicht oder zu wenig zu testen, erkannt wurde.
%Unter anderem, weil es klar/verstanden wurde, wie riskant ist, Testaktivitäten zu überspringen oder nicht genug Testing durchzuführen.

In der Vergangenheit waren Unternehmen gezwungen, ihren Betrieb zu unterbrechen, oder haben aufgrund von Fehlern in ihren Anwendungen Kunden verloren. 

Beispiele dazu sind Starbucks im Jahr 2015, der gezwungen war, etwa 60 \% der Filialen in den USA und Kanada wegen eines Softwarefehlers in seinem Kassensystem zu schließen{\cite{QS1}} oder Nissan im Jahr 2014 als fast 1 Million Fahrzeuge wegen eines Softwarefehlers in den Airbag-Sensoriksystemen zurückgeziehen musste{\cite{QS2}}. 

Neben der Verringerung des Risikos der oben genannten Situationen bietet das Software-Testing weitere Vorteile. 
Zu diesen Vorteilen gehören eine Erhöhung der Softwarequalität, die 
Trennung der zwischen Software-Testing und -entwicklung, bessere Chancen, Fehler zu entdecken, bevor die Kunden sie entdecken und vieles mehr.

Aus diesem Grund wurden Unit-Tests für das Backend und End2End für das Frontend vorgesehen. 
\\

\textbf{Unittest}\\
Auch als Komponententest bekannt, wird diese geschrieben, um die richtige Funktionsweise einer Codeeinheit zu prüfen. In den meisten Programmiersprachen ist das eine Funktion, ein Unterprogramm oder eine Methode.

Für dieses Projekt wurde das Jest Framework verwendet. 
Sowohl Cypress als auch Jest stehen an der Spitze der Liste, wenn es um Test-Frameworks geht{\cite{QS3}}.
%Before/after all; damit der Zustand zurücksetzen kann.
%Ein TF darf nur 1 Sache testen. 
%1 Beispiel erläutern (User)
%Screenshot von Ergebnisse
%Beschreibung von Screenshot: 1. Zeile gibt an
\\
\textbf{End-to-End Testfälle}\\
Cypress IO ist ein Testautomatisierungswerkzeug für End-to-End-Tests und UI-Tests. 
\\
\textbf{Testautomatisierung}\\
Die Automatisierung von Testfällen maximiert die Effizienz, wenn Testfälle in den Kontinuierliche-Integration(Continuous integration) Workflows einbezieht werden. Neuer Code kann automatisch getestet werden, z. B. jedes Mal, wenn ein neues Pull-Request erstellt wird.

\begin{figure}[ht]
	\centering
    \includegraphics[width=\textwidth]{sources/Gitlab-CI.png}\cite{MG10}
	\caption{Gitlab-CI}
	\label{Continuous integration workflow } {\cite{GLAB1}}
\end{figure}


\subsection{Die Testfälle für unser Projekt}
\paragraph{}
Nachstehend einer Überblick über die Testfälle bei Cypress.
\\
\begin{center}
\includegraphics[scale=0.60]{Cy_Test_Cases}\label{fig:Cy_Test_Cases}\\
\textbf{Abbildung \autoref{fig:Cy_Test_Cases}:} Testfälle in Cypress
\end{center}

\begin{center}
\includegraphics[scale=0.40]{Cy_Test_running}\label{fig:Cy_Test_running}
\textbf{Abbildung \autoref{fig:Cy_Test_running}:} Grafische Darstellung der verschiedenen Tests in Cypress
\end{center}


%Ergebnise der ganzen Testlauf

\subsection{Unit test (Backend)}
To-Do



\begin{comment}
Obwohl das Projekt relativ klein ist, wurde die Wichtigkeit von automatisierten Tests nicht unterschätzt.
\\
Für das Frontend wurden End-to-End Testfälle mit Cypress geschrieben.
Auf diese Weise ist es möglich in Sekundenschnelle festzustellen, ob etwas in unserer Anwendung defekt ist.

Ein Szenario, in dem dies hilfreich ist, ist, wenn eine Unterkomponente in anderen Komponenten verwendet wird.

Durch die Änderung der Unterkomponente kann sich diese in einer unerwünschten Weise verhalten.
\\
Das ist der Fall bei der Komponente AvatarImage.
Dies ist eine Funktionskomponente, die 3 Parameter erhält: Größe des Bildes, Bild-URL und Benutzername.

Zu Beginn des Projekts wurde nicht daran gedacht, die Größe des Bildes über einen Parameter dieser Funktion zu steuern. Im Laufe des Projekts wurde uns klar, dass wir die Logik in diesem Element wiederverwenden konnten.
\\
Innerhalb der Komponente wird geprüft, ob eine URL existiert, und wenn ja, wird das mit dem Link verbundene Bild gezeichnet. Falls es keine URL  angegeben wurde, werden die ersten beiden Buchstaben des Benutzernamens verwendet, um ein Standardsymbol zu erzeugen.

In der aktuellen Version des Codes wird diese Komponente in vier anderen Komponenten wieder verwendet.
Wenn das Projekt weiter wachsen würde, würde auch die Möglichkeit von Fehlern im Code zunehmen. Fehler zu finden, wäre in dem Fall aufwändiger.

Ohne automatisierte Tests, ist manuelles Testing nötig.
\\
Im Anhang 2 befindet sich ein Code-Auszug eines Testfälles End-to-End. 
\end{comment}
%%POSIBLEMENTE ESTE NO ES ELMEJOR EJ PUESTO QUE NO SE ESTA PROBANDO EL TAMANIO LAS IMAGENES EN LAS PRUEBAS DE CYPRESS
%% SIN EMBARGO EL HECHO DE REUTILIZAR LOGICA DE CIERTOS ELEMENTOS ES MUY RELEVANTE Y DEBERIA SER INCLUIDA EN OTRA PARTE %DEL REPORTE
%BUSCAR OTRO EJ. DONDE EL TESTING COBRA MAS RELEVANCIA



\newpage
\section*{Zusammenfassung }\label{kap_ZF}
\addcontentsline{toc}{section}{Zusammenfassung}
%Es hat sich gezeigt, dass mit modernen JavaScript-Entwicklungswerkzeugen 
%auch ohne umfassende Kenntnisse der Softwareentwicklung zugänglich sind.
Es hat sich gezeigt, wie die Entwicklung einer Webanwendung nach dem MERN-Stack erfolgt. Dies in einem Zeitraum von Mai 2021 bis Mitte August 2021. Mit grundlegenden Programmierkenntnissen war es möglich, eine funktionelle Anwendung mit folgenden Funktionen zu entwicklen; die Registrierung für neue Benutzer, die Anmeldung für bestehende Benutzer, die Verwaltung deren persönlichen Daten, die Interaktion mit anderen Benutzern auf der Grundlage ihrer Präferenzen und einen auf Textnachrichten basierenden Kommunikationskanal.
\\\\
%TInfra
In Kapitel \ref{kap_Technologieinfrastruktur} wurde die Architektur der Anwendung kurz eingeführt. %DB
Die Auswahl der Datenbank und deren Schemata wurde in Kapitel \ref{kap_Datenbank} erläutert. In Kapitel \ref{kap_Backend} und \ref{kap_Schnittstelle} befindet sich die Logik, um die Daten der Datenbank zu manipulieren. An dieser Stelle wurde erklärt, wie die REST-Schnittstelle für das Profilbild im Amazon S3 und die GraphQL-Endpunkte für die Verwaltung der Nutzerdaten entwickelt wurde.  
%Backend %Schnittstelle 
%Frontend
Das Kapitel \ref{kap_Frontend} zeigt welche Kriterien für die Wahl von React berücksichtigt wurden. Außerdem, wurde die Entwicklung der Benutzeroberfläche mit React-Hooks gezeigt. 
%Qualitätssicherung
Die Erstellung von den Unit- und End-To-End-Tests beschrieben im Kapitel \ref{kap_QS} würde die Testzeit für komplexere Anwendungen beschleunigen. Im Falle des vorliegenden Projekts handelte es sich um ein zusätzlicher Werkzeug, dessen Einbindung in den Entwicklungszyklus etwa fünf Tage in Anspruch nahm. %Einer signifikante Wert haben die Unit-Test, 
%Was konnte nicht geschafft werden?
\\\\
Der Fokus lag daran, eine funktionelle Anwendung zu entwicklen. Es hat sich auf Funktionen verzichtet, welche die geplante Zeitraum des Projekts überschritten. Unter anderem die Anmeldung über SSO mit zum Beispiel eine Facebook- oder Google-Konto. %Für einen späteren Zeitpunkt wird sich dieses Projekt bei WIE HEISST DAS?
Eine bessere Gebrauchlichkeit %Usability 
bietet für die vorliegenden Applikation, eine mobile Applikation. Deshalb ist es sinnvoll eine weitere Version der Anwendung mit React-Native umzuwandeln.

%Glossar
%\newpage


% Anhang
\newpage
\section*{Anhang}\label{anhang}
\addcontentsline{toc}{section}{Anhang} % Manuellen Eintrag im Inhaltsverzeichnis erzeugen
%\appendix
%\chapter{Anexo I: Ejemplo de hoja técnica}
% Contenido del anexo I
%\chapter{Anexo II: Hoja de soluciones}
% Contenido del anexo II

\subsection{Beispiel einer Antwort auf die an den Server gesendete Anfrage }\label{subsec_UabsAnhang}
\begin{lstlisting}
    name: "gilo2754"
    aboutMe: "I like video code video games too"
    avatar: "https://lol-friendfinder-dev.s3.eu-central-1.amazonaws.com/avatars/a028fd00-e767-4dcb-912b-b4828dbe52e7.jpg"
    dateOfBirth: "1990-06-01T00:00:00.000Z"
    gender: "male"
    ingameRole: (2) ["Fill", "Bot"]
    languages: (2) ["bm", "de"]
    friends:(5) ["60eb61b33a2481451c1cd7ac", "60eb61b33a2481451c1cd7b0", "60eb61b33a2481451c1cd7b1", "60eb61b33a2481451c1cd7b4", "60eb61b33a2481451c1cd7b7"]
    blocked: (1) ["60eea5eed7e23b5ef08f68af"]
\end{lstlisting}


\subsection{Beispiel eines End-to-End Testfälles}\label{subsec_UabsAnhang}
\begin{lstlisting}
    //in constants.js sind Standard-Testdaten
    import * as c from "../constants.js"
    
    const qtyLanguages = 101
    const qtyGenders = 8
    
    describe("Check if all profile elements have been rendered", () => {
      it("Login with existing account", () => {
        cy.typeLogin(c.email, c.password)
      })
    
      it("Username was rendered", () => {
        cy.get('#username').should("be.visible")
      })
    
      it("Gender selection was rendered", () => {
        cy.get('#dropdown-gender').should(
          "be.visible")
      })
    
    //Der folgende Test prueft, ob eine Mindestanzahl von Elementen angezeigt wird  
    it(`Gender options is not empty and >= to ${qtyGenders}`, () => {
      cy.get('#dropdown-gender').click()
      cy.get(`.dropdown-menu > :nth-child(${qtyGenders})`).should(
          "be.visible")
      //close the dropdown
      cy.get('#dropdown-gender').click()
    
      })
    ...
    })
    
    \end{lstlisting}
% #3
\subsection{Beispiel eines Commands bei Cypress}\label{subsec_UabsAnhang}
\begin{lstlisting}
    Cypress.Commands.add("typeLogin", (email, password) => {
        cy.visit("/login")
    
        cy.get("[id=email-input]").type(email).should("have.value", email)
    
        cy.get("[id=password-input]").type(password).should("have.value", password)
    
        cy.get("[id=btn-submit]").click()
    })
\end{lstlisting}
Es handelt sich in diesem Fall, um eine benutzerdefinierte Funktion, die zwei Parameter, E-Mail und Passwort, erhält. 
Beide Parameter werden in die entsprechenden Eingabefelder eingetragen, abschließend wird die Eingabetaste gedrückt.
Diese Funktion kann in jedem anderen Testfall aufgerufen werden, ohne dass sie importiert werden muss.

% #4
\subsection{Code-Auszug zum Prüfen und Hochladen von einem Bild mit Axios}\label{subsec_UabsAnhang}
\begin{lstlisting}
    function disableBtn() {
        const uploadBtn = document.getElementById("uploadBtn")
        uploadBtn.disabled = true
        uploadBtn.style.background = "#000000"
      }
    
      function fileUploadHandler() {
        errored ? 
         disableBtn()
         /*
        Tritt ein Fehler auf, wird die Abfrage nicht gesendet und der Knopf deaktiviert 
        Ein Fehler tritt auf, wenn die Datei zu gross ist oder oder wenn sie ein ungueltiges  Format hat
        */
        :
        console.log("uploading pic...", file?.name)
        const fd = new FormData()
        /* avatar ist der Name der Datei, die hochgeladen wird
          file ist der zu sendende  Wert */
        fd.append("avatar", file)  
    
        axios
          .post(urlAvatar, fd, {
            headers: {
              "x-auth-token": TOKEN,
            },
          })
          .then((res) => {
            setState((state) => ({ ...state, avatar: res?.data?.location }))    
            //In location finden wir die URL fuer das gerade hochgeladende Bild 
          })
      }\end{lstlisting}

%#5
\subsection{Definition des Providers für das Teilen von Werten mit dem Consumer }\label{subsec_UabsAnhang}
Dieser Code ermöglicht, die Werte innerhalb von dem Context zu verwenden.
\begin{lstlisting}
    import { React, useState, useEffect, createContext } from "react"

    export const GlobalContext = createContext()
    export default function App() {
        const [token, setToken] = useState()
        const [state, setState] = useState(null)
        
        ...
   
        return (
            <GlobalContext.Provider value={{ token, setToken, state, setState, refetch }}>
              <>
                <MyNavbar />
        
                <Switch>
                  <Route exact path="/" component={Home} />
                  <Route path="/login" component={() => <Login />} />
                  <Route path="/signup" component={SignUp} />
                  <Route exact path="/users" component={() => <Users />} />
                  <Route exact path="/profile" component={() => <Profile />} />
                  <Route exact path="/chat" component={() => <Chat />} />
                  <Route exact path="/ChatMessage" component={() => <ChatMessage />} />
        
                  <Route component={NotFound} />
                </Switch>
              </>
            </GlobalContext.Provider>
          )
        }
//Code-Auszug in frontend/src/App.js
 \end{lstlisting}

%#6
\subsection{GlobalContext, um die Freunde des angemeldeten Benutzers zu zeigen. }\label{subsec_UabsAnhang}
Nachfolgend wird gezeigt, wie GlobalContext in der Komponente Chat.js verwendet wurde, um die Freunde des angemeldeten Benutzers zu ermitteln.

Die Kennungen der Freunde waren notwendig, um die Kommunikation zu ermöglichen und die zwischen den Freunden ausgetauschten Nachrichten zu laden.
\begin{lstlisting}
    import { useContext, useEffect, useState, React } from "react"
    import { GlobalContext } from "../App"
    //mithilfe von Destrukturierende Zuweisung wurden neue Variablen und Funktionen definiert
    const { token, state, setState, refetch } = useContext(GlobalContext)

    ...

    return !token ? (

    ...

        <div className="user-many">
        {state?.friends &&
          state?.friends?.map((item, index) => {
            return (

//Die Komponente FriendList gibt den Namen und den Avatar des Benutzers auf der Grundlage der eingegebenen ID zurueck.
              <FriendList
                setUserID={setUserID}
                setUserNameChat={setUserNameChat}
                setChatAvatar={setChatAvatar}
                userId={item.user}
                searchUser={searchUser}
                setSearchUser={setSearchUser}
                key={index + 1}
              />
            )
          })}
      </div>
    )
  ...
//Code-Auszug in frontend/src/Chat.js
\end{lstlisting}

\subsection{Links zu den verwendeten Technologien und Werkzeugen}\label{subsec_UabsAnhang}
NodeJS
\\
React
\\
Bootstrap
\\
GitHub
\\
Cypress
\\
VSC
\\
Heroku
\\
LaTeX
\\
Cypress

\subsection{Aufwandsverteilung}\label{subsec_UabsAnhang}
Hier zeigen wir, wie die Aufgaben unter den Autoren des Projekts verteilt wurden.
TABELLE/Bild KOMMT...

%Bib
\newpage
\printbibliography[heading=bibintoc]

% Erklärung über die selbständige Abfassung der Arbeit
\pagestyle{empty}
\section*{Erklärung über die selbständige\\Abfassung der Arbeit} % \section*{...}: das *-Symbol erlaubt, dass dieser
% Gliederungspunkt nicht ins Inhaltsverzeichnis aufgenommen wird
\addcontentsline{toc}{section}{Erklärung über die selbständige Abfassung der Arbeit}
Ich versichere, die von mir vorgelegte Arbeit selbständig verfasst zu haben.
Alle Stellen, die wörtlich oder sinngemäß aus veröffentlichten oder nicht veröffentlichten Arbeiten anderer entnommen sind,
habe ich als entnommen kenntlich gemacht.\\
Sämtliche Quellen und Hilfsmittel, die ich für die Arbeit benutzt habe, sind
angegeben. Die Arbeit hat mit gleichem Inhalt bzw. in wesentlichen Teilen noch keiner anderen Prüfungsbehörde vorgelegen.\\\\
\begin{tabular}{cp{7cm}}
                                    &             \\
          Gummersbach                   &             \\ \hline
  \small (Ort, Datum, Unterschrift) & \normalsize \\
\end{tabular}


\newpage
% Unbeschriftetes Abschlussblatt (Leere Seite)
\thispagestyle{empty}
\input{leereSeite}

\end{document}

